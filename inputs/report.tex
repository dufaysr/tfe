%!TEX root = ../tfe.tex
\section*{Test case to assess the implementation of the overturner model}
This note is an adaptation of Eric Deleersnijder's working paper \cite{deleersnijder2011test}.

\subsection*{Governing equations}
Let us consider a water domain, whose width is denoted $A(t,\b{x})$, where $t$ is the time and $\b{x} = (y,z)$ is the position vector. The continuity equation is
\begin{equation} \label{eq:testcase:continuity}
	\frac{\partial A}{\partial t} + \nabla \cdot (A\b{u}) = 0,
\end{equation}
where $\b{u}(t,\b{x})$ is the latitudinally-averaged meridional velocity. Assuming that mixing along the parallels is sufficiently efficient, we may study the concentration of a passive tracer by means of a two-dimensional model. The latitudinally-averaged concentration of the tracer $C(t,\b{x})$ obeys the following partial differential equation :
\begin{equation} \label{eq:testcase:conservative}
	\frac{\partial (AC)}{\partial t} + \nabla \cdot (A \b{u} C) = \sum_{n=1}^N Q_n \delta(\b{x} - \b{x}_n) + \nabla \cdot (A\b{K} \cdot \nabla C), 
\end{equation}
where $\b{K}$ is the diffusivity tensor (symmetric and positive definite); $\delta$ is the Dirac delta function with $\delta(\b{x}-\b{x}_n) = \delta(x-x_n)\delta(y-y_n)$; $Q_n(t)$ is the rate of release of the point source located at $\b{x} = \b{x}_n$. If $C(t,\b x)$ represents the 
density of the tracer in water, then $Q_n(t)$ is the mass of tracer released per second by the $n$-th point source.

Equation \eqref{eq:testcase:conservative} is the so-called conservative form of the model. The convective form is obtained by combining equations \eqref{eq:testcase:continuity} and \eqref{eq:testcase:conservative}:
\begin{equation}  \label{eq:testcase:convective}
	\frac{\partial C}{\partial t} + \b{u} \cdot \nabla C = \frac{1}{A}\sum_{n=1}^N Q_n \delta(\b{x} - \b{x}_n) + \frac{1}{A} \nabla \cdot (A\b{K} \cdot \nabla C).
\end{equation}

\subsection*{An idealised model}
% Recall that we are building a test case for assessing the validity of our implementation of the overturner model.
Now we make some simplifying assumptions which will allow us to build an analytical solution. First, we consider an infinite domain, i.e.
\begin{equation} \label{eq:testcase:domain}
	-\infty < y < \infty, \ -\infty < z < \infty.
\end{equation}
We assume a constant velocity field
\begin{equation} \label{eq:testcase:velocity}
	\b{u}(t,\b{x}) = v \b e_y + w \b e_z,
\end{equation}
where $\b e_y$ and $\b e_z$ are the unit vectors associated respectively with the $y$- and $z$-coordinate axis. Furthermore, the diffusivity tensor is supposed constant and diagonal :
\begin{equation} \label{eq:testcase:diffusivity}
	\b{K} = \begin{pmatrix}
			K_{yy} & 0 \\
			0 & K_{zz}
			\end{pmatrix},	
\end{equation}
where $K_{yy},\ K_{zz} > 0$. Finally, we consider a sudden pointwise release of tracer at $t=0$. Hence, $Q_n(t)$ is of the form :
\begin{equation}
	Q_n(t) = M_n\delta(t), \quad \mbox{ for } 1 \le n \le N.
\end{equation} 

Under these assumptions, equation \eqref{eq:testcase:convective} simplifies to :
\begin{equation} \label{eq:testcase}
	\frac{\partial C}{\partial t} + v \frac{\partial C}{\partial y} + w \frac{\partial C}{\partial z} = \frac{1}{A}\sum_{n=1}^N M_n \delta(t) \delta(y - y_n)\delta(z-z_n) + K_{yy} \frac{\partial^2 C}{\partial y^2} + K_{zz} \frac{\partial^2 C}{\partial z^2}.
\end{equation}

\subsection*{Green's function}
In order to build the analytical solution of the problem, we need to compute the Green's function $\G$ associated to this particular problem. $\G$ is zero for $t<t'$ and is the solution of
\begin{equation}
	\begin{cases}
		\frac{\partial G}{\partial t} + v \frac{\partial G}{\partial y} + w\frac{\partial G}{\partial z} = K_{yy}\frac{\partial^2 G}{\partial y^2} + K_{zz}\frac{\partial^2 G}{\partial z^2}\\[.1cm]
		\left. G(t,t',y,z) \right \rvert_{t=t'} = \delta(y)\delta(z) 
	\end{cases}
\end{equation}
for $t \ge 0$. This yields
\begin{equation} 
	G(t,t',y,z) = \frac{\exp\left[-\frac{(y-s_v)^2}{4K_{yy}\tau} -\frac{(z-s_w)^2}{4K_{zz}\tau} \right]}{4\pi\sqrt{K_{yy}K_{zz}}\tau},
\end{equation}
where $\tau = t-t'$ and 
\begin{equation}
	\b s(t,t') = (s_v(t,t'), s_w(t,t')) = \left(\int_{t'}^{t}v(\xi) \rm d \xi, \int_{t'}^{t}w(\xi) \rm d \xi \right) = \left(v\tau, w\tau\right).
\end{equation}

The Green's function has some interesting properties. The "mass" of the solution is
\begin{equation} \label{eq:testcase:propmass}
	m(t,t') \equiv \int_{\R[2]} \G \rm d \b x = 1.
\end{equation}
The "center of mass" is located at
\begin{equation} \label{eq:testcase:propcenter}
	\b r(t,t') \equiv \frac{1}{m(t,t')} \int_{\R[2]} \b x \G \rm d \b x  = \b s(t,t').
\end{equation}
The variance of the solution is
\begin{equation} \label{eq:testcase:propvar}
	\sigma^2(t,t') \equiv \frac{1}{m(t,t')} \int_{\R[2]} \lvert \b x - \b r(t,t') \rvert^2 \G \rm d \b x = 2 \rm{trace}(\b K) \tau.
\end{equation}

\subsection*{Analytical solution and properties}
The analytical solution of our test case is now obtained with the help of the Green's function derived above :
\begin{equation}
	C(t,\b x) = \int_{t'}^{t} \int_{\R[2]} G(t,t',\b x - \b x') \frac{1}{A} \sum_{n=1}^{N} M_n \delta(t) \delta(\b x- \b x_n) \rm d \b{x}' \rm dt'.
\end{equation}
This expression simplifies to
\begin{equation}
	C(t,\b x) = \frac{1}{A} \sum_{n=1}^{N} M_n G(t,0,\b x - \b x_n).
\end{equation}

The number of tracer's particles present in the domain is
\begin{equation}
	m(t) \equiv A \int_{\R[2]} C(t,\b x) \rm d \b x = \sum_{n=1}^{N} M_n.
\end{equation}
Note that this number is independant of the transport processes.

The mass center is located at
\begin{align}
	\b r(t) &\equiv \frac{A}{m(t)} \int_{\R[2]} \b x C(t,\b x) \rm d \b x \nonumber \\
	&= \frac{1}{\sum_{n=1}^N M_n} \sum_{n=1}^N M_n \int_{\R[2]} \b x G(t,0,\b x - \b x_n) \rm d \b x \nonumber\\
	&= \frac{1}{\sum_{n=1}^N M_n} \sum_{n=1}^N M_n \int_{\R[2]} (\b x-\b x_n) G(t,0,\b x - \b x_n) + \b x_n G(t,0,\b x - \b x_n) \rm d \b x\nonumber\\
	&= \frac{\sum_{n=1}^{N} M_n (\b x_n + \b s(t,0))}{\sum_{n=1}^{N} M_n},
\end{align}
where properties \eqref{eq:testcase:propmass} and \eqref{eq:testcase:propcenter} are used to perform the last step.

Finally, we compute the variance of the solution in the case where there is only one site of pointwise release of tracer. In that case, $N = 1$ and
\begin{align}
	\sigma^2(t) &= \frac{1}{m(t)} \int_{\R[2]} \lvert \b x - \b r(t) \rvert^2 C(t,\b x) \rm d \b x \nonumber \\
	&= \int_{\R[2]} \lvert (\b x - \b x_1) - \b s(t,0)\rvert^2 G(t,0,\b x - \b x_1) \rm d \b x \nonumber \\
	&= 2 \mathrm{trace}(\b K)t \nonumber \\
	&= 2(K_{yy}+K_{zz})t,
\end{align}
where property \eqref{eq:testcase:propvar} is used.
