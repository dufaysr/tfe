%!TEX root = /home/renaud/Documents/EPL/tfe/latex/tfe.tex
\addcontentsline{toc}{section}{List of symbols}
\section*{List of Symbols}
Unless otherwise stated, bold lower cases indicate column vectors ($\b x$) and bold upper cases indicate matrices ($\b X$). The entries of a vector are denoted by a single subscript ($x_i$) and the entries of a matrix are denoted by brackets with a double subscript ($[\b X]_{i,j})$. The transposed sign is $^\t$.

\noindent As a general rule, subscript $t$ indicates a discrete-time quantity evaluated at time $t$, while parenthesis indicate a continuous-time quantity.

\noindent The time derivative of a quantity $x$ is denoted by the usual convention $\partial x/\partial t$, or by $\dot{x}$.
\subsection*{Clustering}
\begin{table}[H]
\begin{tabular}{ll}
	$n$ & Number of nodes of the graph.\\
	$\b A$ & Weighted adjacency matrix.\\
	$q_i$ & Total outgoing weight from node $i$ ($i \in \{1,\dots,n\}$).\\
	$\b q$ & Vector of the $q_i$'s.\\
	$\b Q$ & Matrix with the total outgoing weights on its diagonal: $\b Q = \diag(\b q)$.\\
	$\b M$ & Transition probability matrix for the discrete-time Markov process.\\
	$\b P(t)$ & Transition matrix for the continuous-time Markov process evaluated at time $t$.\\
	$p_{t,i}$ & Probability to be in node $i$ at Markov time $t$ ($i \in \{1,\dots,n\}$).\\
	$\b p_t$ & Discrete-time probability vector: $\b p_t = [p_{t,1},\dots,p_{t,n}]$.\\
	$\b p(t)$ & Continuous-time probability vector.\\
	$\bs \pi$ & Stationary distribution vector.\\
	$\bs \Pi$ & Matrix with the stationary distributions on its diagonal: $\bs \Pi = \diag(\bs \pi)$.\\
	$\b x_t$ & Random indicator vector at time $t$.\\
	$\b C(\b x_{t_0},\b x_{t_0+t})$ & Autocovariance matrix of $\b x$.\\
	$\P$ & A partition.\\
	$c$ & Number of communities.\\
	$\C_i$ & The $i$th community of the partition ($i \in \{1,\dots,c\}$).\\
	$\b H_{\P}$ & Indicator matrix of $\P$.\\
	$\mathcal{H}_{\P}(\cdot)$ & Operator $\b H_\P^\t (\cdot) \b H_\P$.\\
	$\b y_t$ & Community indicator vector at time $t$.\\
	$\b R_t(\P)$ & Discrete-time clustered autocovariance matrix for partition $\P$ at time $t$.\\
	$\b R(t;\P)$ & Continuous-time clustered autocovariance matrix for partition $\P$ at time $t$.\\
	$r_t(\P)$ & Discrete-time stability of partition $\P$.\\
	$r(t;\P)$ & Continuous-time stability of partition $\P$.\\
	$r_t$ & Discrete-time stability curve.\\
	$r(t)$ & Continuous-time stability curve.\\
	$\tau$ & Teleportation probability.\\
	$\lambda_i(\b q)$ & Rate at which continuous-time random walkers leave node $i$ ($i \in \{1,\dots,c\}$).\\
	$\bs \lambda(\b q)$ & Vector of the $\lambda_i(\b q)$'s.\\
	$\b L$ & Laplacian matrix.\\
	$\VI(\P_1,\P_2)$ & Variation of information between partitions $\P_1$ and $\P_2$.
\end{tabular}
\end{table}
\newpage
\subsection*{Transport model}
Adimensional variables are denoted by primes.
\begin{table}[H]
\begin{tabular}{ll}
	$C$ & Concentration function.\\
	$C(\b x,t)$ & Concentration function with its dependence on position and time written\\ & explicitly.\\
	$q$ & Source/sink term, or reactive term.\\
	$\b u$ & Velocity vector.\\
	$\b K$ & Diffusivity tensor (symmetric and positive-definite).\\
	$\b x$ & Position vector. In two dimensions, $\b x = (y,z)$.\\
	$\rho$ & Density of the mixture (seawater).\\
	$\rho_w$ & Density of pure water.\\
	$\Omega$ & Domain of the problem (time independent).\\
	$\partial \Omega$ & Boundary of $\Omega$.\\
	$|\Omega|$ & Volume (area in two dimensions) of $\Omega$.\\
	$\n$ & Outward unit normal vector.\\
	$[\cdot]_{*}$ & Quantity $\cdot$ evaluated at $*$.\\
	$\bar{C}(t)$ & Mean concentration over $\Omega$ that possibly depends on $t$.\\
	$\bar{C}$ & Mean concentration over $\Omega$, shown to be constant.\\
	$\hat{C}(\b x,t)$ & Deviation of the concentration with respect to $\bar{C}$.\\
	$\sigma^2(t)$ & Variance of the concentration function.\\
\end{tabular}
\end{table}

\subsection*{Compartment model}
In what follows, $i,j \in \{1,\dots,N\}$.
\begin{table}[H]
\begin{tabular}{ll}
	$N$ & Number of subdomains (or compartments).\\
	$\Omega_i$ & Subdomain $i$.\\
	$|\Omega_i|$ & Volume (area in two dimensions) of $\Omega_i$.\\
	$\Gamma_{i,j}$ & Interface between $\Omega_i$ and $\Omega_j$.\\
	$\Gamma_{i,e}$ & Interface between $\Omega_i$ and the environment.\\
	$|\Gamma_{i,j}|$ & Area (length in two dimensions) of $\Gamma_{i,j}$.\\
	$C_i(t)$ & Concentration function for compartment $i$.\\
	$q_i(t)$ & Net production rate over $\Omega_i$.\\
	$U_{i,j}$ & Advective "flux" from compartment $i$ to compartment $j$.\\
	$V_{i,j}$ & Diffusive "flux" from compartment $i$ to compartment $j$.\\
	$\sigma_{box}^2(t)$ & Box variance.\\
	$\b c$ & Vector of the compartment's concentrations: $\b c = [C_1(t),\dots,C_N(t)]^\t$.\\
	$\b q$ & Vector of the $q_i$'s: $\b q = [q_1(t),\dots,q_N(t)]^\t$.\\
	$\bs \omega$ & Vector of the $|\Omega_i|$'s: $\bs \omega = [|\Omega_1|,\dots,|\Omega_N|]^\t$.\\
	$\bs \Omega$ & Matrix with $\bs \omega$ on its diagonal: $\bs \Omega = \diag (\bs \omega)$.\\
	$\b A$ & Interaction matrix.\\
	$\b A_{\Delta t}$ & Discrete interaction matrix for time step $\Delta t$.\\
	$P_i(t)$ & Number of tracer's particles in compartment $i$ at time $t$.\\
	$P$ & Mass of one tracer's particle.\\
\end{tabular}
\end{table}

\subsection*{SDEs}
\begin{table}[H]
\begin{tabular}{ll}
	$\delta(t)$ & Dirac delta function.\\
	$\xi(t)$ & White noise.\\
	$W(t)$ & Wiener process.\\
	$\I$ & Indicate an Itô SDE or integral.\\
	$\bI$ & Indicate a backward Itô SDE or integral.\\
	$\esp{\cdot}$ & Expectation.\\
	$p$ & Probability density function.\\
	$p(x,t;y,s)$ & Probability density function with its dependence on position and time,\\& and the initial condition written explicitly.\\
	$X_i$ & Numerical approximation of $x(t_i)$.\\
	$\Delta W_i$ & $W(t_i)-W(t_{i-1})$.\\
	$\Delta t_i$ & $t_i-t_{i-1}$.\\
\end{tabular}
\end{table}

\subsection*{From clusters to compartments}
\begin{table}[H]
\begin{tabular}{ll}
	$\nby$ & Number of grid cells in the $y$-direction.\\
	$\nbz$ & Number of grid cells in the $z$-direction.\\
	$N_{cell}$ & Total number of grid cells.\\
	$m_{i,j}(T)$ & Probability that a particle ends up in grid cell $j$ after a time $T$\\ & if it was initially in grid cell $i$.\\
	$\b M(T)$ & Adjacency matrix of the graph at time $T$, which is also the transition\\ & probability matrix.\\
	$P_{i \rightarrow j}(T)$ & Number of particles in grid cell $j$ at time $T$, which were initially in grid cell $i$.\\
	$P_0$ & Number of particles released in each grid cell.\\
\end{tabular}
\end{table}
