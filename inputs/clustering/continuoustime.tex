%!TEX root = /home/renaud/Documents/EPL/tfe/latex/tfe.tex
\section{Extension to continuous time}
From a general viewpoint, the discrete process can be interpreted as an approximation of its continuous counterpart: whereas the state of the discrete-time random walker can only change at unit-time intervals, the continuous-time random walkers undergo a waiting time between each change of state which is itself a random variable. More precisely, the waiting time is a continuous memoryless random variable distributed exponentially. Obviously, the transition probabilities from one node to the other are the same for both discrete- and continuous-time processes, only the time at which the jump occurs may vary. The continuous-time process corresponding to~\eqref{eq:discreteMP} is governed by the following dynamics:
\begin{equation} \label{eq:continuousMP_general}
    	\dot{\b p} = \b p \diag\left\{\bs \lambda(\b q)\right\} \b Q^{-1} \b A - \b p \diag\left\{\bs \lambda(\b q)\right\} = -\b p \b L,
\end{equation}
where $\lambda_i(\b q)$ is the rate at which random walkers leave node $i$, and $\b L = \diag\{\bs \lambda(\b q)\}[-\b Q^{-1} \b A + \b I]$. Two particular cases of this process are implemented by the stability software and are thus examined here, depending on the choice of $\bs \lambda(\b q)$: the so-called \textit{normalized Laplacian dynamics} and \textit{standard (combinatorial) Laplacian dynamics}. Their names come from the similarity that arises between $\b L$ and the normalized/standard Laplacian matrix. Each of those two dynamics represent best different physical processes. The former corresponds to the choice $\bs \lambda_{norm}(\b q) = \b 1$. Hence, the expected waiting time is $1$ at every node, and $\b L = -\b Q^{-1} \b A + \b I = -\b M + \b I$. The latter corresponds to $\bs \lambda_{combi}(\b q) = \b q/\langle \b q \rangle$. In that case, $\b L = (-\b A + \b Q)/\langle \b q \rangle$ and the average waiting time at node $i$ is $\langle \b q \rangle/q_i$. Hence, the expected waiting time at a given node is smaller (resp. larger) than $1$ if the total weight of the outgoing edges from that node is larger (resp. smaller) than the average total weight of the outgoing edges on the network. However, the expected waiting time over the whole network is $\langle \langle \b q \rangle/\b q \rangle = 1$. The corresponding governing equations are respectively 
\begin{equation} \label{eq:continuousMP_norm}
	\dot{\b p} = \b p \b Q^{-1} \b A - \b p = \b p \b M - \b p
\end{equation}
for the normalized Laplacian and
\begin{equation} \label{eq:continuousMP_combi}
    	\dot{\b p} = \b p \frac{\b A}{\langle \b q \rangle} - \b p\frac{\b Q}{\langle \b q \rangle}
\end{equation}
for the combinatorial Laplacian.

The clustered autocovariance matrix for partition $\P$ at time $t$ is easily generalized to 
\begin{equation}
	\b R(t;\P) = \b H_{\P}^{\t}(\b \Pi\b P(t) - \bs \pi^{\t}\bs \pi)\b H_{\P},
\end{equation}
where $\b P(t)$ is the the transition matrix of the process at time $t$: $\b P(t) = \Exp^{-t\b L}$. The continuous-time definition of the stability of a partition $\P$ follows almost straightforwardly:
\begin{equation}
	r(t;\P) = \trace \left[ \b R(t; \P) \right].
\end{equation}
Notice that it is not necessary to minimize over the time interval $[0,t]$: indeed, it can be shown that $\trace \left[ \b R(t;\P) \right]$ is monotonically decreasing with time. The interpretation in terms of a random walk is similar to the discrete case: let $P(\C,t)$ be the probability that a random walker is in community $\C$ at time $t$ if it was initially in $\C$, when the system is at stationarity. Discounting the probability of such an event to take place by chance at stationarity and summing over all communities of $\P$ leads to the definition of the stability of the partition $\P$:
\begin{equation} \label{eq:generalstability}
	r(t;\P) = \sum_{\C \in \P} P(\C,t) - P(\C,\infty).
\end{equation}
By ergodicity, the memory of the initial condition is lost at infinity and $P(\C,\infty)$ is thus equal to the probability that two independent walkers are in $\C$ at stationarity. Equation~\eqref{eq:generalstability} tells us that only the communities in which a random walker is likely to stay bring a positive contribution to stability, where \textit{likely to stay} means that the probability for a walker to be in its initial community at time $t$ is larger than the probability of that event occurring by chance at stationarity. The stability curve of the graph can now be expressed as a continuous function of $t$:
\begin{equation}
	r(t) = \max_{\P} r(t;\P).
\end{equation}