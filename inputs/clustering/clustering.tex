%!TEX root = /home/renaud/Documents/EPL/tfe/latex/tfe.tex
\chapter{The stability criterion for community detection} \label{chap:clustering}
% \section{The stability criterion for graph communities} \label{sec:stability}
The partition of a graph into communities (or clusters) has been widely studied those last two decades. Clustering comes indeed pretty handy to gain insight into the underlying structure of a system represented by a network. In some cases one can even build a simplified functional description of the system based on the clusters. Many partitioning methods have been proposed, each relying on a particular measure to quantify the quality of a community structure. Such methods include normalized cut, ($\alpha$,$\epsilon$) clustering or modularity and its variants and extensions. The reader may refer to \cite{fortunato2010community} for a 2010 survey of the different clustering methods. In this work, we choose the stability approach, which is based on the statistical properties of a dynamical process taking place on the network. This approach was initially presented in \cite{delvenne2010stability} and further expended in \cite{lambiotte2009laplacian} and \cite{delvenne2013stability}. 

The stability method presents a number of advantages. First, it does not require the number of communities to be specified beforehand, ensuring a natural partitioning of the graph. Second, it is flexible in the sense that it does not seek a \textit{unique} optimal partition. Instead, it reveals several community structures, each appearing to be the most relevant at particular values of the Markov time: at a given time scale, natural clusters corresponds to sets of states from which escape is unlikely within that time scale. The stability method provides thus a dynamical interpretation of the partitioning problem. The Markov time acts as an intrinsic resolution parameter, as will be developed shortly. Finally, it is probably the most unifying approach since many of the standard partitioning measures find an interpretation through the stability framework.

In order to compute stability partitions in the next of this work, we make use of Michael Schaub's free software \textit{PartitionStability}. This C++ implementation of the stability method with a \matlab interface is available at \url{https://github.com/michaelschaub/PartitionStability}. It relies on the Louvain algorithm \cite{blondel2008fast} to optimize the stability quality function. This heuristic algorithm has been initially developed for modularity optimization. However one can show that stability can be written as the \textit{modularity} of a time-dependent network evolving under the Markov process \cite{lambiotte2009laplacian}. Hence, the Louvain method can almost straightforwardly be applied to stability optimization.

This chapter is devoted to the explanation of the stability measure, and how to find good clusterings using stability analysis. It acts as a theoretical part intended to cover everything that is needed to make a proper, informed use of the stability toolbox. Notice that the stability measure has initially been presented for discrete times in \cite{delvenne2010stability}. We follow the same approach here: discrete-time stability is developed in the first section of this chapter; it is then extended to continuous time in a second section; finally, a few tools to analyze the robustness of a partition are presented in the third section of the chapter.
%!TEX root = /home/renaud/Documents/EPL/tfe/latex/tfe.tex
\section{Discrete-time stability as an autocovariance}
The stability criterion is based on the two-way relationship between graphs and Markov chains: On one hand, any graph has an associated Markov chain where the states are the nodes of the graph and the transitions probabilities between states are given by the weights of the edges. On the other hand, any Markov chain can be represented by a graph whose edges are weighted according to the transition probabilities. Concretely, consider a graph of $n$ nodes whose $n \times n$ weighted adjacency matrix is denoted $\b A$. Let $\b q = \b A \b 1$; $q_i$ is thus the total weight of the outgoing edges from node $i$. Let $\b Q = \mathrm{diag}(\b q)$. Then, by normalizing the rows of $\b A$ we get the matrix $\b M = \b Q^{-1}\b A$, the transition probability matrix. $\b M$ is row-stochastic (or right-stochastic) and $[\b M]_{ij}$ is the probability to go from node $i$ to node $j$. 
Consider a particle moving in the network according to the transition probabilities in $\b M$. Now let $\b p_t$ be the $1 \times n$ probability vector at Markov time $t$, namely that $p_{t,i}$ is the probability that the particle is located in node $i$ at time $t$. The dynamics of the discrete-time Markov process are given by :
\begin{equation} \label{eq:discreteMP}
	\b p_{t+1} = \b p_t \b K^{-1}\b A = \b p_t \b M.  	
\end{equation} 
Now, suppose that the Markov chain is ergodic, i.e. that it is possible to go from every state to every state and that the Markov process is aperiodic. The ergodicity assumption implies that any initial state will asymptotically reach the same stationary solution. Let $\bs \pi$ be that stationary distribution, obtained by solving $\bs \pi = \bs \pi \b M$, and $\b \Pi = \rm{diag}(\bs \pi)$. Now, let $\b x_t$ be the $n$-dimensional random indicator vector describing the position of a particle undergoing the above dynamics: $x_{t,i} = 1$ if the particle is located in node $i$ at time $t$, and $0$ otherwise. At stationarity, the \textit{autocovariance matrix} of $\b x$ is
\begin{subequations}
	\begin{align}
 	\b C(\b x_{t_0},\b x_{t_0+t}) &\triangleq \E\left[(\b x_{t_0} - \E[\b x_{t_0}])^{\t}(\b x_{t_0+t} - \E[\b x_{t_0+t}])\right] \\
 		&= \E\left[(\b x_{t_0} - \bs \pi)^{\t}(\b x_{t_0+t} - \bs \pi)\right] \\
 		&= \E\left[\b x_{t_0}^{\t} \b x_{t_0+t}\right] - \E[\b x_{t_0}^{\t}] \bs \pi - \bs \pi^{\t} \E\left[\b x_{t_0+t}\right] + \bs \pi^{\t} \bs \pi \\
 		&= \bs \Pi \b M^t - \bs \pi^{\t} \bs \pi,
 	\end{align}
\end{subequations}
where the fact that $\b C(\b x_{t_0},\b x_{t_0+t})$ only depends on the time difference $t$ at stationarity is readily verified. Here, $^\t$ is the transposed sign and $\b M^t$ is $\b M$ at the power $t$. $[\b C(\b x_{t_0},\b x_{t_0+t})]_{ij}$ is interpreted as the correlation between $\b x_{t_0,\,i}$ and $\b x_{t_0+t,\,j}$. The independence on the initial time $t_0$ implies that it can indifferently be chosen equal to $0$.

Suppose now a partition $\P$; we note $\b H_{\P}$ the indicator matrix of $\P$. If $c$ is the number of communities in $\P$, $\b H_{\P}$ is a binary $n \times c$ matrix such that 
\begin{equation}
	[\b H_{\P}]_{ik} = 
	\begin{cases}
		1  & \quad \mbox{if node $i$ is in community $k$},\\
	    0  & \quad \text{otherwise}.\\
	\end{cases}
\end{equation}
Let us define $\mathcal{H}_{\P} : \R[n \times n] \rightarrow \R[c \times c] : \b B \mapsto \mathcal{H}_\P(\b B) = \b H_{\P}^{\t} \b B \b H_{\P}$. Let $\b X$ be any $n \times n$ matrix, then $\b Y = \mathcal{H}_{\P}(\b X)$ is a $c \times c$ matrix such that 
\begin{equation}
	[\b Y]_{kl} = \sum_{i \in \C_k} \sum_{j \in \C_l} [\b X]_{ij},	
\end{equation}
where $\C_k$ and $\C_l$ denote communities $k$ and $l$ of partition $\P$. One could thus say that operator $\mathcal{H}_{\P}$ returns the \textit{clustered version} of any $n \times n$ matrix, namely the matrix where the contributions of every nodes belonging to the same community are gathered by summing them. Finally, let $\b y_t = \b H_{\P}^{\t} \b x_t$ denote the $c$-dimensional community indicator vector: $\b y_{t,\,k}$ is equal to $1$ if the particle is in community $k$ at time $t$ and zero otherwise.
Using those notations and the interpretation of $\mathcal{H}_{\P} $, the \textit{clustered autocovariance matrix} for partition $\P$ at time $t$ is defined as
\begin{subequations}
	\begin{align}
		\b R_t(\P) &= \mathcal{H}_{\P}\left(\b C(\b x_{t_0},\b x_{t_0+t})\right)\\
			&= \b C(\b y_{t_0},\b y_{t_0+t})\\
			&= \b H_{\P}^{\t}(\b \Pi\b M^t - \bs \pi^{\t}\bs \pi)\b H_{\P}.
	\end{align}
\end{subequations}
Notice that $\b R_t$ depends only on the topology of the graph and on the partition. If the graph has well defined communities given by $\P$ \textit{over a given time scale}, we expect that the particle is more likely to remain within the starting community over that time scale. This implies that the values of $\b y_{0,\,i}$ and $\b y_{t,\,i}$ are positively correlated for $t$ in that time scale, which in turn implies large diagonal elements in $\b R_t(\P)$ and hence a large trace of $\b R_t(\P)$. The elements of $\b R_t(\P)$ are interpreted as follows in terms of the random walk of a particle: $[\b R_t(\P)]_{kl}$ is the probability that a particle is in community $\C_l$ after $t$ discrete time-steps if it has started in $\C_k$ minus the probability that two independent random walkers are in $\C_k$ and $\C_l$, evaluated at stationarity. A good partition is such that there is a high likelihood of remaining in the starting community over a given time scale. The definition of the stability of a \textit{clustering} $\P$ follows naturally:
\begin{equation}
	r_t(\P) = \min_{0 \le s \le t} \sum_{i = 1}^{c} [\b R_s]_{ii} = \min_{0 \le s \le t} \trace(\b R_s).
\end{equation}
Note that taking the minimum for all times up to $t$ implies that the stability of the clustering at time $t$ is large only if it is large for all times preceding $t$. This allows to assign a low stability to partitions where there is a high probability of leaving the community and coming back to it later. According to \cite{delvenne2013stability}, this minimization is unnecessary in most cases and we have $r_t(\P) \approx \trace(\b R_t)$. Nevertheless, taking the minimization ensures maximum generality and allows for example to deal with almost bipartite graphs where $\trace(\b R_s)$ can be oscillatory.

All the definitions introduced until now are for a given partition $\P$. But what we ultimately want to compute is the optimal partition in the sense of stability, hence the one that maximizes the stability measure. Clearly, the optimal partition might be different for each Markov time $t$. Computing the optimal clustering for each Markov time gives the \textit{stability curve of the graph} :
\begin{equation}
	r_t = \max_{\P} r_t(\P).
\end{equation}
Now we understand how Markov time acts as an intrinsic resolution parameter: as Markov time grows, the number of communities is expected to decrease, since there are more possibilities for a random walker to escape a community when the time window increases. Hence, communities get bigger (or coarser) as Markov time increases. Interestingly, one can prove that in the case of \textit{undirected} networks, stability at time 1 is equivalent to the well-known \textit{configuration modularity} measure. But this equivalence does not hold for \textit{directed} networks and therefore does not concern the present work.

At this stage, an important remark has to be made about the assumption of ergodicity. The verification of this assumption is often far from being obvious, especially in the case of big undirected networks. The trick in that case is to introduce "à la Google" random teleportations.\footnote{In the original PageRank proposed by S. Brin and L. Page in 1998 (ref. \cite{grin1998anatomy}), this consists essentially in applying a perturbation to the transition probability matrix between web pages in order to ensure that at least one row of the matrix is positive, which implies the convergence of the Power Method. If we note the teleportation probability $\tau$, the perturbation can be interpreted as follows: a web surfer follows a link in his current page with probability $1-\tau$ and jumps to an arbitrary web page with probability $\tau$.} Let $\tau$ be the \textit{teleportation probability}. Then, if a random walker is located on a node with at least one outlink (which is always the case for the networks that we will consider), it follows one of the outlinks with probability $1-\tau$. Otherwise, the node is called a \textit{dangling node} and the random walker is teleported with a uniform probability to another random node. The corresponding perturbation of the transition probability matrix is, in the most general case:
\begin{equation} \label{eq:M_teleport}
	\widetilde{\b M} = (1-\tau)\b M + \frac{1}{n}[(1-\tau)\b d + \tau \b 1]\b 1^{\t},
\end{equation}
where $n$ is the number of nodes, $\b d$ is a binary $n \times 1$ vector whose entries are equal to $1$ if the corresponding node is a dangling node and $0$ otherwise, and $\b 1$ is the $n \times 1$ unity vector. In the case that we will consider in the next section, $\b d$ is the zero vector. This perturbation is known to make the dynamics ergodic, ensuring the existence and uniqueness of the stationary solution $\bs \pi$.
%!TEX root = /home/renaud/Documents/EPL/tfe/latex/tfe.tex
\section{Extension to continuous time}
From a general viewpoint, the discrete process can be interpreted as an approximation of its continuous counterpart: whereas the state of the discrete-time random walker can only change at unit-time intervals, the continuous-time random walkers undergo a waiting time between each change of state which is itself a random variable. More precisely, the waiting time is a continuous memoryless random variable distributed exponentially. Obviously, the transition probabilities from one node to the other are the same for both discrete- and continuous-time processes, only the time at which the jump occurs may vary. The continuous-time process corresponding to~\eqref{eq:discreteMP} is governed by the following dynamics:
\begin{equation} \label{eq:continuousMP_general}
    	\dot{\b p} = \b p \diag\left\{\bs \lambda(\b q)\right\} \b Q^{-1} \b A - \b p \diag\left\{\bs \lambda(\b q)\right\} = -\b p \b L,
\end{equation}
where $\lambda_i(\b q)$ is the rate at which random walkers leave node $i$, and $\b L = \diag\{\bs \lambda(\b q)\}[-\b Q^{-1} \b A + \b I]$. Two particular cases of this process are implemented by the stability software and are thus examined here, depending on the choice of $\bs \lambda(\b q)$: the so-called \textit{normalized Laplacian dynamics} and \textit{standard (combinatorial) Laplacian dynamics}. Their names come from the similarity that arises between $\b L$ and the normalized/standard Laplacian matrix. Each of those two dynamics represent best different physical processes. The former corresponds to the choice $\bs \lambda_{norm}(\b q) = \b 1$. Hence, the expected waiting time is $1$ at every node, and $\b L = -\b Q^{-1} \b A + \b I = -\b M + \b I$. The latter corresponds to $\bs \lambda_{combi}(\b q) = \b q/\langle \b q \rangle$. In that case, $\b L = (-\b A + \b Q)/\langle \b q \rangle$ and the average waiting time at node $i$ is $\langle \b q \rangle/q_i$. Hence, the expected waiting time at a given node is smaller (resp. larger) than $1$ if the total weight of the outgoing edges from that node is larger (resp. smaller) than the average total weight of the outgoing edges on the network. However, the expected waiting time over the whole network is $\langle \langle \b q \rangle/\b q \rangle = 1$. The corresponding governing equations are respectively 
\begin{equation} \label{eq:continuousMP_norm}
	\dot{\b p} = \b p \b Q^{-1} \b A - \b p = \b p \b M - \b p
\end{equation}
for the normalized Laplacian and
\begin{equation} \label{eq:continuousMP_combi}
    	\dot{\b p} = \b p \frac{\b A}{\langle \b q \rangle} - \b p\frac{\b Q}{\langle \b q \rangle}
\end{equation}
for the combinatorial Laplacian.

The clustered autocovariance matrix for partition $\P$ at time $t$ is easily generalized to 
\begin{equation}
	\b R(t;\P) = \b H_{\P}^{\t}(\b \Pi\b P(t) - \bs \pi^{\t}\bs \pi)\b H_{\P},
\end{equation}
where $\b P(t)$ is the the transition matrix of the process at time $t$: $\b P(t) = \Exp^{-t\b L}$. The continuous-time definition of the stability of a partition $\P$ follows almost straightforwardly:
\begin{equation}
	r(t;\P) = \trace \left[ \b R(t; \P) \right].
\end{equation}
Notice that it is not necessary to minimize over the time interval $[0,t]$: indeed, it can be shown that $\trace \left[ \b R(t;\P) \right]$ is monotonically decreasing with time. The interpretation in terms of a random walk is similar to the discrete case: let $P(\C,t)$ be the probability that a random walker is in community $\C$ at time $t$ if it was initially in $\C$, when the system is at stationarity. Discounting the probability of such an event to take place by chance at stationarity and summing over all communities of $\P$ leads to the definition of the stability of the partition $\P$:
\begin{equation} \label{eq:generalstability}
	r(t;\P) = \sum_{\C \in \P} P(\C,t) - P(\C,\infty).
\end{equation}
By ergodicity, the memory of the initial condition is lost at infinity and $P(\C,\infty)$ is thus equal to the probability that two independent walkers are in $\C$ at stationarity. Equation~\eqref{eq:generalstability} tells us that only the communities in which a random walker is likely to stay bring a positive contribution to stability, where \textit{likely to stay} means that the probability for a walker to be in its initial community at time $t$ is larger than the probability of that event occurring by chance at stationarity. The stability curve of the graph can now be expressed as a continuous function of $t$:
\begin{equation}
	r(t) = \max_{\P} r(t;\P).
\end{equation}
%!TEX root = /home/renaud/Documents/EPL/tfe/latex/tfe.tex
\section{Assessing the robustness of a partition} \label{subsec:robustness}
We present here two mechanisms commonly used to assess the relevance of a particular partition. One simple way is to consider that a robust partition should not be altered by a small modification of the quality function. Such a modification could be for example a perturbation of the Markov time $t$ at which the partition has been found. From this point of view, robust partitions correspond to \textit{plateaux} in the community curve of the graph. In other words, robust partitions should be persistent over a wide interval of Markov time.

\begin{sloppypar} 
The second indicator of the robustness of a partition that we will take into account in this work follows from considering that a robust partition is one that is persistent to small modifications of the optimization algorithm. The central tool to quantify this approach of the robustness of a partition is the \textit{normalized variation of information} \cite{meilua2007comparing}, which is a popular way to compare two partitions. Let $p(\C)$ be the probability for a node to be in community $\C$, i.e. $p(\C) = n_\C/n$ where $n_\C$ is the number of nodes in community $\C$. The variation of information between partitions $\P_1$ and $\P_2$ is defined as
\begin{equation} \label{eq:clustering_VI}
	\VI(\P_1,\P_2) := \frac{H(\P_1,\P_2)-H(\P_1)-H(\P_2)}{\log (n)} = \frac{H(\P_1|\P_2)+H(\P_2|\P_1)}{\log (n)},
\end{equation}
where $\log(n)$ is a normalization factor; 
\begin{equation}
	H(\P) := -\sum_{\C \in \P} p(\C) \log[p(\C)] 	
\end{equation}
is the Shannon entropy; $H(\P_1,\P_2)$ is the Shannon entropy of the joint probability $p(\C_1,\C_2)$ that a node belongs both to a community $\C_1$ of $\P_1$ and to a community $\C_2$ of $\P_2$. We have 
\begin{equation}
	p(\C_1,\C_2) = \frac{n_{\C_1 \cap\, \C_2}}{n},	
\end{equation}
and
\begin{equation}
	H(\P_1,\P_2) := -\sum_{\C_1 \in \P_1} \sum_{\C_2 \in \P_2} p(\C_1,\C_2) \log[p(\C_1,\C_2)].
\end{equation}
Similarly, $H(\P_1|\P_2)$ is the conditional Shannon entropy of partition $\P_1$ given $\P_2$, which is defined in a standard way from the joint distribution: $p(\C_1|\C_2) = p(\C_1,\C_2)/p(\C_2) = n_{\C_1\cap\, \C_2}/n_{\C_2}$, and the expression of $H(\P_1|\P_2)$ follows straightforwardly. The latter can be interpreted as the additional information needed to describe $\P_1$ once $\P_2$ is known. This measure of the difference between two partitions is then used as follows: for each Markov time, an ensemble of Louvain optimizations of stability are performed, starting from different random initial node ordering.\footnote{Remember that the problem being $\mathcal{NP}$-hard, we rely on a heuristic algorithm --- the Louvain method --- that finds a good partition for a given Markov time, but not necessarily the optimal partition. Hence the partition found may differ if a different initial condition is provided.} The normalized variation of information allows then to quantify how different the optimized partitions are. Therefore, a low variation of information indicates optimized partitions that are very similar to each others, and thus that a small modification of the algorithm barely alter the partition. From the point of view of the field of dynamical system, robust partitions have thus an attractor with a large basin of attraction for the optimization method. 
\end{sloppypar}