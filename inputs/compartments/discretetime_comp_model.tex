%!TEX root = /home/renaud/Documents/EPL/tfe/latex/tfe.tex
\section{Discrete-time compartment model} \label{sec:dtcm(chapcomp)}
It is also possible to build a discrete-time compartment model, and this could even be more relevant than a continuous-time one in the context of numerical implementation. Consider equation~\eqref{eq:compartmentmatrix}. For a given time step $\Delta t > 0$, we would like to build a similar expression that would link $\b c(t+\Delta t)$ to $\b c(t)$. From the theory of ordinary differential equations~\cite{EDO}, we know that the general solution to that equation is
\begin{equation} \label{eq:generalODEsol}
	\b c(t) = \Exp^{\bs \Omega^{-1}\b A(t-t_0)} \b c(t_0) + \int_{t_0}^t \Exp^{\bs \Omega^{-1}\b A(t-s)} \b q(s) \rm ds.
\end{equation}
For a general source term $\b q$, it is not possible to write a matrix expression of the form $\b c(t+\Delta t) = \b A_{\Delta t} \b c(t) + \b B_{\Delta t} \b q(t)$, but it is possible under certain conditions on $\b q$. In this work, we won't need such conditions because we make the assumption that there is no source/sink term, i.e. we assume that $\b q = 0$. Let $t_0 < t_1 < t_2 < \dots$ be such that $t_{k+1} = t_k + \Delta t$. If $\b c(t_k)$ is known, $\b c(t_{k+1})$ can be computed using~\eqref{eq:generalODEsol} with $\b q = 0$:
\begin{equation}
	\b c(t_{k+1}) = \Exp^{\bs \Omega^{-1}\b A\Delta t} \b c(t_k).
\end{equation}
Doing this, we have found a matrix relation that links $\b c(t+\Delta t)$ to $\b c(t)$:
\begin{equation} \label{eq:generaldiscretecompartment}
	\b c(t + \Delta t) = \b A_{\Delta t} \b c(t),
\end{equation}
where $\b A_{\Delta t} = \Exp^{\bs \Omega^{-1}\b A\Delta t}$ is the \textit{discrete interaction matrix}. Notice that the discrete interaction matrix is closely related to the transition probability matrix introduced in chapter~\ref{chap:clustering}. Indeed, developing~\eqref{eq:generaldiscretecompartment} for $C_i(t)$ yields
\begin{equation} \label{eq:crapaud}
	C_i(t+\Delta t) = [\b A_{\Delta t}]_{i,1} C_{1}(t) + [\b A_{\Delta t}]_{i,2} C_{2}(t) + \ldots + [\b A_{\Delta t}]_{i,N} C_{N}(t).  	
\end{equation}
Let $P_i(t)$ denote the number of tracer's particles in compartment $i$ at time $t$, and let $P$ be the mass of one tracer's particle. Notice that
\begin{equation}
	C_i(t) = \frac{P_i(t) P}{|\Omega_i|}.
\end{equation}
Multiplying equation~\eqref{eq:crapaud} by $|\Omega_i|/P$ yields after some manipulations
\begin{equation}
	P_i(t) = [\b A_{\Delta t}]_{i,1} \frac{|\Omega_i|}{|\Omega_1|} P_{1}(t) + [\b A_{\Delta t}]_{i,2}\frac{|\Omega_i|}{|\Omega_2|} P_{2}(t) + \ldots + [\b A_{\Delta t}]_{i,N}\frac{|\Omega_i|}{|\Omega_N|} P_{N}(t).
\end{equation}
Hence, the factor $\frac{|\Omega_i|}{|\Omega_j|}[\b A_{\Delta t}]_{i,j}$ can be interpreted as the probability \label{page:probability_interpretation} for a tracer's particle to end up in compartment $i$ after a period $\Delta t$ if it was initially in compartment $j$. This consideration suggests that the entries of $\b A_{\Delta t}$ should be nonnegative: this is indeed an important property of the discrete interaction matrix that shall be proven shortly.
\begin{property} \label{prop1_discr_comp}
	The entries of the discrete interaction matrix are nonnegative:
	\begin{equation}
		[\b A_{\Delta t}]_{ij} \ge 0 \mbox{ for every } i,j = 1,\dots,N.
	\end{equation}
\end{property}
\begin{proof}
	Let $C_0 > 0$ be a positive constant, and fix $j \in \{1,\dots,N\}$. Suppose that 
	\begin{equation}
		\b c(t) = \begin{pmatrix} c_1(t) \\ \vdots \\ c_{j-1}(t) \\ c_j(t) \\ c_{j+1}(t) \\ \vdots \\ c_N(t) \end{pmatrix}
				= \begin{pmatrix} 0 \\ \vdots \\ 0 \\ C_0 \\ 0 \\ \vdots \\ 0 \end{pmatrix}.
	\end{equation}
	By equation~\eqref{eq:generaldiscretecompartment},
	\begin{equation}
		\b c(t+\Delta t) = \b A_{\Delta t} \b c(t) = \begin{pmatrix} [\b A_{\Delta t}]_{1,j} C_0 \\ \vdots \\ [\b A_{\Delta t}]_{N,j} C_0 \end{pmatrix}.
	\end{equation}
	Since $\b c$ is a concentration, we must have that $\b c(t) \ge \b 0$ at any time $t$. Therefore, since $C_0 > 0$ we must have that
	\begin{equation}
		[\b A_{\Delta t}]_{i,j} \ge 0 \mbox{ for every } i = 1,\dots,N.
	\end{equation}
	Finally, as we have made no assumption on $j$, this must be true for any $j \in \{1,\dots,N\}$, which concludes the proof.
\end{proof}
The interpretation of $\frac{|\Omega_i|}{|\Omega_j|}[\b A_{\Delta t}]_{i,j}$ in terms of a transition probability of the tracer's particles between compartments suggests another important property of the discrete interaction matrix. Indeed, the tracer's particles can neither disappear nor be created, hence the sum of the probabilities over all possible destinations (i.e. over all compartments since we consider an isolated domain) must be equal to one:
\begin{equation}
	\sum_{i=1}^N \frac{|\Omega_i|}{|\Omega_j|}[\b A_{\Delta t}]_{i,j} = 1 \quad \mbox{for every } j=1,\dots,N.
\end{equation}
This property can be deduced from property~\ref{prop1_comp}:

\begin{property} \label{prop1ter_comp} \label{prop2_discr_comp}
	The columns of the discrete interaction matrix satisfy the following relation:
	\begin{equation}
		\sum_{i=1}^N |\Omega_i|  [\b A_{\Delta t}]_{ij} = |\Omega_j| \quad \mbox{for every } j=1,\dots,N.
	\end{equation}
	In matrix form:
	\begin{equation}
		\bs \omega^\t \b A_{\Delta t} = \bs \omega^\t .
	\end{equation}
\end{property}
\begin{proof}
	The expression of the mean concentration over $\Omega$ is the same as in the continuous case:
	\begin{equation}
		\bar C = \frac{1}{|\Omega|} \b 1^\t \bs \Omega \b c.
	\end{equation}
	Since $\bar C$ is constant, we must have that
	\begin{equation}
		|\Omega| \bar C(t+\Delta t) = |\Omega| \bar C(t) = \b1^\t \bs \Omega \b c(t) = \bs \omega^\t \b c(t),
	\end{equation}
	but by equation~\eqref{eq:generaldiscretecompartment} we also have that
	\begin{equation}
		|\Omega| \bar C(t+\Delta t) = \b1^\t \bs \Omega \b c(t+\Delta t) = \b1^\t \bs \Omega \b A_{\Delta t} \b c(t) = \bs \omega^\t \b A_{\Delta t} \b c(t).
	\end{equation}
	Hence, $\bs \omega^\t \b c(t) = \bs \omega^\t \b A_{\Delta t} \b c(t)$, and since this must be true for every possible value of $\b c(t)$, we must have that
	\begin{equation} \label{eq:prop1ter_comp}
		\bs \omega^\t = \bs \omega^\t \b A_{\Delta t},
	\end{equation}
	the desired result.
\end{proof}
\begin{corollary}{prop1ter_comp} \label{corollary2}
	If the compartments all have the same size, the discrete interaction matrix is  left stochastic:
	\begin{equation}
		\b 1^\t \b A_{\Delta t} = \b 1^\t.
	\end{equation}
\end{corollary}
\begin{proof}
	Let $|\Omega_0|$ be the size of the compartments. Then, $\bs \omega = |\Omega_0| \b 1$ and equation~\eqref{eq:prop1ter_comp} reduces to
	\begin{equation}
		|\Omega_0| \b 1^\t = |\Omega_0| \b 1^\t \b A_{\Delta t}.
	\end{equation}
\end{proof}
Property~\ref{prop2_comp} also has an interpretation in terms of the discrete interaction matrix:
\begin{property} \label{prop2ter_comp} \label{prop3_discr_comp}
	The discrete interaction matrix is right stochastic:
	\begin{equation}
		\b A_{\Delta t} \b 1 = \b 1.
	\end{equation}
\end{property}
\begin{proof}
	Let $C_0 > 0$ be a constant and $\b c(t) = C_0 \b 1$. Introducing the latter in~\eqref{eq:generaldiscretecompartment} yields
	\begin{equation}
		C_0 \b 1 = \b A_{\Delta t} C_0 \b 1 = C_0 \b A_{\Delta t} \b 1.
	\end{equation}
\end{proof}
Two last properties allow to bound the entries of the discrete interaction matrix:
\begin{property} \label{prop4_discr_comp}
	The entries of the discrete interaction matrix are smaller than one:
	\begin{equation}
	 		[\b A_{\Delta t}]_{i,j} \le 1 \mbox{ for every } i,j = 1,\dots,N.
	\end{equation}
\end{property}
\begin{proof}
	We proceed by contradiction. Suppose there exists $i,j \in \{1,\dots,N\}$ such that $[\b A_{\Delta t}]_{i,j} > 1$. By property~\ref{prop1_discr_comp}, all the entries of $\b A_{\Delta t}$ are nonnegative so that
	\begin{equation}
		\sum_{k = 1}^N [\b A_{\Delta t}]_{i,k} > 1,
	\end{equation}
	in contradiction with property~\ref{prop2ter_comp}.
\end{proof}
\begin{property} \label{prop5_discr_comp}
	The entries of the discrete interaction matrix satisfy
	\begin{equation}
		\frac{|\Omega_i|}{|\Omega_j|} [\b A_{\Delta t}]_{i,j} \le 1 \mbox{ for every } i,j = 1,\dots,N.
	\end{equation}
\end{property}
\begin{proof}
	We proceed again by contradiction. Suppose there exist $i,j \in \{1,\dots,N\}$ such that $\frac{|\Omega_i|}{|\Omega_j|} [\b A_{\Delta t}]_{i,j} > 1$. By property~\ref{prop1_discr_comp}, all the entries of $\b A_{\Delta t}$ are nonnegative so that
	\begin{equation}
		\sum_{k = 1}^N \frac{|\Omega_k|}{|\Omega_j|}[\b A_{\Delta t}]_{k,j} > 1,
	\end{equation}
	in contradiction with property~\ref{prop1ter_comp}.
\end{proof}
This last property is in agreement with the interpretation of the factors $\frac{|\Omega_i|}{|\Omega_j|} [\b A_{\Delta t}]_{i,j}$ as probabilities of transition between the compartments.