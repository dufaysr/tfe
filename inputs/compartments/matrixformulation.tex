%!TEX root = /home/renaud/Documents/EPL/tfe/latex/tfe.tex
\section{Matrix formulation}
The generic compartment model described in the previous sections can be written in matrix form. For simplicity and because we will restrict ourselves to such systems in the next, we consider the case of an isolated system, i.e. $\phi_{i,e} = 0$. Let
\begin{equation} \label{eq:def_compartment_vars}
	\b c = \begin{pmatrix} C_1 \\ C_2 \\ \vdots \\ C_N \end{pmatrix}, \quad \b q = \begin{pmatrix} q_1 \\ q_2 \\ \vdots \\ q_N \end{pmatrix}, \quad \bs \omega = \begin{pmatrix} \Omega_1 \\ \Omega_2 \\ \vdots \\ \Omega_N \end{pmatrix}, \quad \mbox{and} \quad \bs \Omega = \diag(\bs \omega).
\end{equation}
The evolution of the concentrations in the compartments is given by
\begin{equation} \label{eq:compartmentmatrix}
	\bs \Omega \b{\dot{c}} = \bs \Omega \b q + \b A \b c,
\end{equation}
where $\b A$ is the \textit{interaction matrix}, which describes the advective and diffusive fluxes between the subdomains. Using the parameters introduced previously, $\b A$ can be expressed as
\begin{equation} \label{eq:generalA}
	\b A = 	\begin{pmatrix}
				-\sum_{j=1}^N \frac{1}{2}U_{1,j}+V_{1,j} & -\frac{1}{2}U_{1,2}+V_{1,2} & \cdots & -\frac{1}{2}U_{1,N}+V_{1,N}\\[.1cm]
				-\frac{1}{2}U_{2,1}+V_{2,1} & -\sum_{j=1}^N \frac{1}{2}U_{2,j}+V_{2,j} & \cdots & -\frac{1}{2}U_{2,N}+V_{2,N}\\
				\vdots & \vdots & \ddots & \vdots\\
				-\frac{1}{2}U_{N,1}+V_{N,1} & -\frac{1}{2}U_{N,2}+V_{N,2} & \cdots & -\sum_{j=1}^N \frac{1}{2}U_{N,j}+V_{N,j}
			\end{pmatrix}.
\end{equation}
In matrix formulation, the mean concentration is expressed as
\begin{equation} \label{eq:Cmeanmatrix}
	\bar C = \frac{1}{\Omega} \b 1^\t \bs \Omega \b c,
\end{equation}
where $\b 1$ is the N-dimensional unit column vector. Notice that $\Omega = \b 1^\t \bs \Omega \b 1$. In the next of this section, we show of properties~\ref{prop1_comp}, \ref{prop2_comp} and~\ref{prop3_comp} from section~\ref{sec:prop_comp} are expressed in matrix formulation.
\begin{propertybis}{prop1_comp} \label{prop1bis_comp}
Each column of the interaction matrix sums to zero:
\begin{equation}
	\b 1^\t \b A = 0.
\end{equation}
\end{propertybis} 
\begin{proof}
	We know that $\dot{\bar C} = 0$. From equations~\eqref{eq:compartmentmatrix} and~\eqref{eq:Cmeanmatrix},
	\begin{equation}
		\Omega \dot{\bar{C}} = \b 1^\t \bs \Omega \dot{\b{c}} = \b 1^\t \bs \Omega \b q + \b 1^\t\b A \b c.
	\end{equation}
	Furthermore, equation~\eqref{eq:prop1deleersnijder} in an isolated system is equivalent to
	\begin{equation}
		\Omega \dot{\bar{C}} = \b 1^\t \bs \Omega \b q,
	\end{equation}
	hence we must have that $\b 1^\t \b A = 0$.
\end{proof}

\begin{propertybis}{prop2_comp} \label{prop2bis_comp}
	Each row of the interaction matrix sums to zero:
	\begin{equation}
		\b A \b 1 = 0.		
	\end{equation}	
\end{propertybis}
\begin{proof}
	Let $C_0$ be a constant and $\b c = C_0 \b 1$. Introducing the latter in~\eqref{eq:compartmentmatrix} with $\b q = \b 0$ yields
	\begin{equation}
		0 = C_0 \b A \b 1.
	\end{equation}
\end{proof}

\begin{propertybis}{prop3_comp} \label{prop3bis_comp}
	The interaction matrix is negative definite.
\end{propertybis}
\begin{proof}
	We consider a passive tracer hence $\b q = 0$. Besides, since the domain is isolated the domain-averaged value of the concentration $\bar C$ is a constant.
	Consider $\hat{\b c} = \b c - \bar C \b 1$, the deviation of the concentration with respect to $\bar C$. Since under those conditions,~\eqref{eq:compartmentmatrix} is linear and homogeneous in $\b c$, it applies to any perturbation of $\b c$ by a constant, and thus to $\hat{\b c}$:
	\begin{equation} \label{eq:prp3matrix}
		\bs \Omega \dot{\hat{\b c}} = \b A \hat{\b c}.
	\end{equation}
	The variance is expressed as
	\begin{equation}\label{eq:sigmamatrix}
		\sigma^2 = \frac{1}{\Omega} \hat{\b c}^\t \bs \Omega \hat{\b c}.
	\end{equation}
	A time-differentiation of~\eqref{eq:sigmamatrix} yields
	\begin{equation}
		\frac{d \sigma^2}{dt} = \frac{2}{\Omega} \hat{\b c}^\t \bs \Omega \dot{\hat{\b c}},
	\end{equation}
	and finally by~\eqref{eq:prp3matrix}
	\begin{equation}
		\frac{\Omega}{2} \frac{d \sigma^2}{dt} = \hat{\b c}^\t \b A \hat{\b c}.
	\end{equation}
	The latter must be negative (or zero if $\hat{\b c} = \b 0$), which shows that $\b A$ is negative definite.
\end{proof}