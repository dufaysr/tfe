%!TEX root = /home/renaud/Documents/EPL/tfe/latex/tfe.tex
\section{Properties of the compartment model solution} \label{sec:prop_comp}
In section~\ref{sec:propcontinuous}, important properties of the continuous transport model have been derived. In this section, the counterparts of those properties are established in the case of a compartment model. The domain-averaged concentration can be expressed in terms of the compartments concentrations. Indeed, by equations \eqref{eq:Cmean} and \eqref{eq:C_i(t)}:
\begin{equation} \label{eq:Cmean_compartment}
	\bar{C}(t) = \frac{1}{|\Omega|} \int_\Omega C(\b x,t) \rm d\Omega = \frac{1}{|\Omega|} \sum_{i=1}^N \int_{\Omega_i} C(\b x,t) \rm d\Omega_i
		= \frac{1}{|\Omega|} \sum_{i=1}^N |\Omega_i| C_i(t).
\end{equation}
The variance of the concentration is approximated by the \textit{box variance}
\begin{equation}
	\sigma_{box}^2(t) = \frac{1}{|\Omega|} \sum_{i=1}^{N} |\Omega_i| \left[C_i(t) - \bar{C}(t)\right]^2.
\end{equation}

\begin{property} \label{prop1_comp}
	The mean tracer concentration $\bar C$ is not influenced by the internal advective and diffusive fluxes.
\end{property}
\begin{proof}
	By~\eqref{eq:Cmean_compartment},~\eqref{eq:compartmentmodel},~\eqref{eq:Uprop} and~\eqref{eq:Vprop}, we get successively
	\begin{subequations}
	\begin{align}
		\frac{d\bar{C}}{dt} &= \frac{1}{|\Omega|} \sum_{i=1}^N |\Omega_i| \frac{dC_i(t)}{dt}\\
			&= \frac{1}{|\Omega|} \sum_{i=1}^N \left[|\Omega_i| q_i - \sum_{j=1}^N \left[ U_{i,j}\frac{C_i + C_j}{2} - V_{i,j} (C_j-C_i)\right] - \phi_{i,e} \right]\\
			&= \frac{1}{|\Omega|} \sum_{i=1}^N(|\Omega_i| q_i + \phi_{e,i}). \label{eq:prop1deleersnijder}
	\end{align}
	\end{subequations}
\end{proof}

\begin{corollary}{prop1_comp} \label{cor:Cmeanconstant}
	For a passive tracer in an isolated domain, the mean tracer concentration within the whole domain $\bar C$ is a constant.
\end{corollary}
\begin{proof}
	The proof is straightforward by introducing $q_i = 0$ (passive tracer) and $\phi_{i,e} = 0$ (isolated domain) in equation~\eqref{eq:prop1deleersnijder}.	
\end{proof}

\begin{property} \label{prop2_comp}
	If the tracer is passive, the domain isolated and the initial concentration is constant, then the concentration stays constant.
\end{property}
\begin{proof}
	If the initial concentration is constant, then $C_i(0) = C_0$ for all $i = 1,\dots,N$. At time $t=0$, equation~\eqref{eq:compartmentmodel} becomes
	\begin{equation}
		|\Omega_i| \frac{dC_i}{dt} = -C_0 \sum_{j=1}^N U_{i,j},
	\end{equation}
	which is equal to zero by~\eqref{eq:continuitycompartment}. Here we have used that $q_i = 0$ and $\phi_{i,e} = 0$ (and thus $U_{i,e}=0$) since the tracer is passive and the domain is isolated.
\end{proof}

\begin{property} \label{prop3_comp}
	For a passive tracer ($q_i = 0$) in an isolated domain ($\phi_{i,e} = 0$), the box variance of the concentration decreases monotonically until $C_i(t) = \bar{C}(0)$ for all $i = 1,\dots,N$ (i.e., until the tracer distribution is uniform and equal to $\bar C(0)$ everywhere).
\end{property}
\begin{proof}
	Let $\hat{C}_i(t)$ denote the deviation of the concentration in compartment $i$ with respect to the mean concentration over the whole domain:
	\begin{equation}
		\hat{C}_i(t) := C_i(t) - \bar C(t).
	\end{equation}
	By property~\ref{prop2_comp}, $\bar C(t) = \bar C(0) = \bar C$ since the domain is isolated and the tracer is passive. Hence, we can rewrite equation~\eqref{eq:compartmentmodel} as 
	\begin{equation}
		|\Omega_i| \frac{d\hat C_i}{dt} = - \sum_{j=1}^N \left[ U_{i,j}\frac{\hat C_i + \hat C_j}{2} - V_{i,j} (\hat C_j - \hat C_i)\right] - \bar C \sum_{j=1}^N U_{i,j},
	\end{equation}
	which by~\eqref{eq:continuitycompartment} simplifies to
	\begin{equation} \label{eq:temp_comp1}
		|\Omega_i| \frac{d\hat C_i}{dt} = - \sum_{j=1}^N \left[ U_{i,j}\frac{\hat C_i + \hat C_j}{2} - V_{i,j} (\hat C_j - \hat C_i)\right].
	\end{equation}
	Multiplying~\eqref{eq:temp_comp1} by $\hat C_i(t)$ and summing over $i = 1,\dots,N$ yields after some calculations
	\begin{equation} \label{eq:temp_comp2}
		|\Omega| \frac{d \sigma^2_{box}}{dt} = - \sum_{i=1}^N \sum_{j=1}^N \left[ U_{i,j}(\hat C_i + \hat C_j)\hat C_i - 2 V_{i,j} (\hat C_j - \hat C_i) \hat C_i\right].
	\end{equation}
	Let us look at both terms in the right-hand side of~\eqref{eq:temp_comp2} separately. The first term is equal to zero:
	\begin{subequations}
	 	\begin{align}
	 		\sum_{i=1}^N \sum_{j=1}^N U_{i,j}(\hat C_i + \hat C_j)\hat C_i &= \sum_{i=1}^N \left(\hat C_i \right)^2 \underbrace{\sum_{j=1}^N U_{i,j}}_{= 0} + \sum_{i=1}^N \sum_{j=1}^N U_{i,j}\hat C_j\hat C_i\\
	 		&= \sum_{i=1}^N \sum_{j=1}^N \underbrace{\frac{U_{i,j} + U_{j,i}}{2}}_{=0}\hat C_j\hat C_i.
	 	\end{align}
	 \end{subequations} 
	The second term is rewritten as
	\begin{equation}
 		2 \sum_{i=1}^N \sum_{j=1}^N V_{i,j} (\hat C_j - \hat C_i) \hat C_i = -2 \sum_{i=1}^N \sum_{j=1}^N V_{i,j} (\hat C_j - \hat C_i)^2 - 2\sum_{i=1}^N \sum_{j=1}^N V_{i,j} (\hat C_j - \hat C_i) \hat C_i,
	 \end{equation} 
	 so that
	 \begin{equation}
	 	2 \sum_{i=1}^N \sum_{j=1}^N V_{i,j} (\hat C_j - \hat C_i) \hat C_i = - \sum_{i=1}^N \sum_{j=1}^N V_{i,j} (\hat C_j - \hat C_i)^2.
	 \end{equation}
	 Finally,
	 \begin{equation}
	 	|\Omega| \frac{d \sigma^2_{box}}{dt} = - \sum_{i=1}^N \sum_{j=1}^N V_{i,j} (\hat C_j - \hat C_i)^2,
	 \end{equation}
	 and thus
	 \begin{equation}
	 	\frac{d \sigma^2_{box}}{dt} = -\frac{1}{|\Omega|} \sum_{i=1}^N \sum_{j=1}^N V_{i,j} (C_j - C_i)^2,
	 \end{equation}
	 which proves the claim.  Interestingly, the advective part of the fluxes does not contribute to the homogenization of the concentration; only the diffusive part does.
\end{proof}