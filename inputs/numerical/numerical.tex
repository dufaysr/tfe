%!TEX root = /home/renaud/Documents/EPL/tfe/latex/tfe.tex
\chapter{Numerical Considerations} \label{chap:numerical}
Recall that the final goal of this work is to develop a method to automatically delineate subdomains that would be relevant compartments in a box model for a geophysical flow problem, and then to develop and assess compartment models based on the so calculated subdomains. The idea is to determine the boxes via the community detection method presented in chapter~\ref{chap:clustering}. In this work, we focus on \textit{two-dimensional} problems involving \textit{passive} tracers. The evolution of the concentration of such tracers is dictated by the advection-diffusion equation~\eqref{eq:C_PDE_vec}. In order to apply the community detection method, we need to emphasize the tracer's problem as a graph problem, and we aim to compute the transition probability matrix $\b M(t)$ at some wisely chosen times $t$ on that graph. To this end, the solution to the tracer's transport model~\eqref{eq:C_PDE_vec} must be computed at different times and for different initial conditions. However, the advection-diffusion equation can in general not be solved analytically: the goal of this chapter is thus to provide all the material that is necessary to develop efficient numerical methods for the tracer's transport problem. A lot of different numerical methods have been developed through the years, each having their pros and cons. In the next paragraphs, we briefly summarize the main families of methods available, with their main advantages and disadvantages. This allows then to make an informed decision about which method to use in this work. Most of the discussion in the following paragraphs is inspired from \cite{spivakovskaya2007lagrangian} and \cite{spivakovskaya2007backward}. 

A very popular class of numerical methods is formed by the Eulerian methods, in which the advection-diffusion equation is solved on a fixed spatial grid. This class encompass the finite difference method, finite element method and finite volume method. A second class is formed by the Lagrangian methods, where particles are followed through space at every time step. As we shall see later, the movement of an individual particle is modeled with a stochastic differential equation (SDE) which is consistent with the advection-diffusion equation. The idea is to estimate the concentration by simulating the trajectories of a large number of particles and taking averages. Several averaging methods have been developed to estimate the concentration from the set of particles positions: we shall come back to this further in section~\ref{boxcounting_kernel}. Finally, a third class of mixed Eulerian-Lagrangian methods has been developed, which basically attempts to combine the advantages of both approaches. Such conceptually attractive methods have been widely applied in applications. They are however more complex to implement; in view of the relative simplicity of the two-dimensional tracer's transport problem studied in this work, we will not consider such mixed methods here.

Both Eulerian and Lagrangian methods have their own advantages and disadvantages. The Eulerian methods provide the convenience of a fixed grid and are easy to implement. The main drawbacks are the inherent dispersion errors and artificial oscillations, leading to solutions that may be neither mass conservative nor positive \cite{stijn1987positive}, \cite{yang1998accuracy}. Basically, the effect of dispersion errors is similar to physical dispersion but it is due to truncation errors. Artificial oscillations are typical from higher order methods designed to reduce dispersion errors. Those downsides could become excessively severe in case of problems involving a sharp concentration front, for instance advection-dominated problems or problems with large gradients on the initial concentration field (typically delta-like initial concentration) \cite{zheng2002applied}. In those cases, numerical dispersion (i.e. dispersion due to dispersion errors) tends to inappropriately smooth out the sharp concentration front, whereas artificial oscillations tends to become more important, leading to serious problems with the positiveness of the solution. 

On the other hand, Lagrangian methods ensures that the solution is always mass conservative and nonnegative. They are thus more suited than Eulerian methods to advection-dominated problems and to problems with large concentration gradients, since they do not suffer from dispersion errors and artificial oscillations. Moreover, if the tracer does not occupy the whole domain, the Lagrangian methods may be computationally more efficient than their Eulerian counterpart \cite{hunter1987application}. Depending on the number of particles used, Lagrangian methods may also require less storage than finite differences or finite element methods. Another advantage of Lagrangian methods is that they make it possible to advect the particles exactly when  the velocity field can locally be described by an analytical function \cite{hunter1993use}. Finally, because each realization of the particle movement is independent from the others, Lagrangian methods are perfect candidates for parallelization. For instance, the MPI library makes it pretty easy and efficient to parallelize a code based on random walk models. Among the drawbacks of particle methods, the lack of a fixed grid may lead to numerical instability and computational difficulties \cite{yeh1990lagrangian}. If flow variables are not known analytically, their interpolation to the particle location could lead to local mass balance errors and solution anomalies \cite{labolle1996random}. Finally, the number of particles needed to get a smooth solution might be large leading to a large computational time, but this can be compensated by a parallelization of the code.

Considering the above discussion, it seems that a Lagrangian method is more appropriate for our concern. Indeed, in order to build an approximation of the transition probability matrix, the domain $\Omega$ is partitioned into grid cells which are the nodes of the graph. For a simulation time $T$, an entry $[\b M(T)]_{ij}$ is the probability that a particle goes from grid cell $i$ to grid cell $j$ in a time $T$. To estimate the entries of a line $i$ of the matrix, the idea is to run a simulation for a time $T$ and an initial concentration which is uniform in grid cell $i$ and zero in every other grid cell. The initial concentration is thus sharp, a first argument in favor of a Lagrangian method. Furthermore, the concentration is interpreted as a probability, hence positiveness and mass conservation are crucial topics, another point for Lagrangian methods. Finally, the flow variables are known analytically in the problems that will be considered in this work, which considerably reduce the drawbacks of Lagrangian methods pointed out above. For these reasons, we choose to go for a Lagrangian method. 
% Concentration initiale sharp
% concentration interprétée comme une proba -> positiveness & mass conservation ++
% flow variables known analytically : dans ce cas la plupart des désavantages exprimés ci-dessus sont sans objet
%!TEX root = /home/renaud/Documents/EPL/tfe/latex/tfe.tex
\section{Preliminaries}
We introduce here the notions of stochastic differential equations (SDE's) and stochastic integrals. Those are the fundamental tools at the basis of the Lagrangian numerical methods. The idea behind such methods is to estimate the concentration obeying an advection-diffusion-reaction equation by simulating the trajectories of a large number of particles in the flow. In this work, we restrict ourselves to advection diffusion equations of the form \eqref{eq:C_PDE_vec}, which we recall here for the sake of readability:
\begin{equation} \label{eq:ADE}
	\frac{\partial \b C}{\partial t} = \nabla \cdot (-\b u C + \b K \nabla C)
\end{equation}
In the next, equation \eqref{eq:ADE} will be referred to as the \textit{transport model}. In order to implement a numerical method tracking the fates of individual particles, an equation describing the fate of such a particle must be derived, and that equation must be consistent with the transport model. Formally, the transport model can be interpreted as a Fokker-Planck equation, namely the partial differential equation governing the time evolution of the probability density function $p(\b x,t)$ of the position of a particle. The correspondence is made by interpreting the concentration as the probability density function: $p=C$. 

At the microscopic scale, Brownian diffusion is modeled by a stochastic force acting on the particles. This force is interpreted as the resultant of atomic bombardment on the particle. Intuitively, the direction of the force due to atomic bombardment is constantly changing, and at different times the particle is hit more on one side than another, leading to the seemingly random nature of the force, and hence of the motion. Therefore, the differential equation governing the position $\b x(t)$ of a particle is stochastic. For example, Langevin proposed in 1908 an equation governing the position of a Brownian particle, which in 1D can be written in the form :
\begin{equation} \label{eq:Langevin}
	\frac{dx}{dt} = a(x,t) + b(x,t)\xi(t),
\end{equation}
where $x$ is the position of the particle, $a(x,t)$ and $b(x,t)$ are known functions and $\xi(t)$ is the rapidly fluctuating random term. More precisely, $\xi(t)$ is a white noise, i.e.
\begin{subnumcases}{}
		\esp{\xi(t)} = 0,\\
		\esp{\xi(t)\xi(t')} = \delta(t-t'),
\end{subnumcases}
where $\esp{\cdot}$ denotes expectation. The fact that $\xi$ has zero mean is because any nonzero mean can be absorbed in the term $a(x,t)$. The second condition states that $\xi(t)$ is uncorrelated, namely that the random force acting on a particle at a time is independent of the random forces acting on that particle at any other time. This simple form of the noise is of course an unrealistic idealization. It is possible to show that
\begin{equation} \label{eq:whitenoise_integral}
	\int_0^t \xi(t') \rm dt' = W(t),
\end{equation}
where $W(t)$ is the \textit{Wiener process}, a stochastic process defined by the following characteristics:
\begin{subnumcases}{} \label{eq:WienerProcess}
	W(0) = 0,\\
	W(t_2) - W(t_1)  \sim \mathcal{N}(0,t_2-t_1),\\
	\esp{[W(t_4)-W(t_3)][W(t_2)-W(t_1)]} = 0, \label{eq:independent_inc}
\end{subnumcases}
where $t_1 < t_2 < t_3 < t_4$. In other words, $W(t)$ is a zero mean gaussian process of variance $t$ which has the property of independent increments. Suppose now that $a(x,t) = a$ and $b(x,t) = b$ are constant. The above relations imply that the solution to $\eqref{eq:Langevin}$ is
\begin{equation}
	x(t) = at + bW(t),
\end{equation}
where we implicitly assumed that $x(0) = 0$. However, one can show that the Wiener process is not differentiable with probability 1 (see for example \textcolor{red}{blabla}). We are thus faced with a paradox here since this implies that $x(t)$ is itself non-differentiable, and hence that the Langevin equation as stated in \eqref{eq:Langevin} \textit{does not exist mathematically}. In fact, $\xi(t)$ is the derivative of $W(t)$ in the \textit{distributive sense}. From \eqref{eq:whitenoise_integral}, it follows directly that
\begin{equation}
	\rm dW(t) \equiv W(t+\rm dt) - W(t) = \xi(t) \rm dt,
\end{equation}
but it is incorrect (or at least very misleading) to write $\frac{dW(t)}{dt} = \xi(t)$, since the Wiener process is nowhere differentiable with probability 1, as already stated above.

Hopefully this introductory example shows the need for some preliminary steps in order to rigorously define a SDE, and to formalize the link between SDE's and Fokker-Planck equations. This is precisely the goal of this section.

\subsection{Formal definition of a SDE}
In this section, we restrict ourselves to a 1-dimensional problem. This allows to make the notations less cumbersome while still introducing all the tools and concepts that are needed in order to understand the 2-dimensional Lagrangian model which is at the basis of our numerical resolution of the overturner model for the meridian concentration in the Atlantic ocean. Indeed, all the results presented here can almost straightforwardly be extended to several dimensions. Consider again the Langevin equation \eqref{eq:Langevin}. We have shown in the introduction that the equation does not really make sense under that form. What we are now going to show is that this corresponding \textit{integral equation} 
\begin{equation}
	x(t) = x(0) + \int_0^t a(x(s),s) \rm ds + \int_0^t b(x(s),s) \xi(s) \rm ds
\end{equation}
can be interpreted consistently. 
Consider the SDE
\begin{equation}
	\left\{
	\begin{array}{rcl} \label{eq:generalSDE}
		\rm dX_t &=& f(X_t,t) \rm dt + g(X_t,t) \rm dW_t\\
		X_{t_0} &=& x_0. 
	\end{array}
	\right.
\end{equation}

\subsection{Fokker-Plank equation for a Backward-Itô SDE}
Consider a particle whose 1D position $X_t$ obeys the SDE \eqref{eq:generalSDE}, considered in the backward-Itô sense. For a small $\Delta t$, we have
\begin{equation}
	X_{t+\Delta t} = X_t + f(X_t,t) \Delta t + g(X_{t+\Delta t},t+\Delta t)[W_{t+\Delta t}-W_t] + \dots
\end{equation}
Let $\Delta W_t := [W_{t+\Delta t}-W_t]$ It is a gaussian random variable of mean $0$ and variance $\Delta t$: $\Delta W_t \sim \mathcal{N}(0,\Delta t)$. 

%!TEX root = /home/renaud/Documents/EPL/tfe/latex/tfe.tex
\section{Lagrangian equations corresponding to the advection-diffusion transport model}
The previous section introduces all the theoretical tools needed to compute the Lagrangian equations corresponding to the transport model, namely the general advection-diffusion equation
\begin{equation} \label{eq:TM}
	\frac{\partial C}{\partial t} = \nabla \cdot (-\b u C + \b K \nabla C),
\end{equation}
where $\b K$ is the symmetric and positive-definite diffusivity tensor.
From now on, we restrict ourselves to two dimensions in the cartesian coordinate system $(y,z)$. 
The transport equation~\eqref{eq:TM} can be interpreted as a Fokker-Planck equation where $C$ is the probability density function of the position $\b x(t) = (y(t),\,z(t))$ of the particle. Let
\begin{equation}
	\b K = \begin{pmatrix} K_{yy} & K_{yz}\\ K_{zy} & K_{zz} \end{pmatrix},
\end{equation}
with $K_{yz} = K_{zy}$ since $\b K$ is symmetric. In order to get the Itô and backward-Itô systems of SDEs corresponding to the transport model, we must rewrite~\eqref{eq:TM} in the forms~\eqref{eq:FP-I-ndim} and~\eqref{eq:FP-bI-ndim} respectively. The systems of SDEs can then be deduced straightforwardly by analogy.

Let us first compute the Itô SDEs corresponding to~\eqref{eq:TM}. Equation~\eqref{eq:TM} can be rewritten as
\begin{multline}
	\frac{\partial C}{\partial t} = -\frac{\partial}{\partial y}\left[\left(v+\frac{\partial K_{yy}}{\partial y} + \frac{\partial K_{yz}}{\partial z}\right)C\right] -\frac{\partial}{\partial z}\left[\left(w+\frac{\partial K_{zy}}{\partial y} + \frac{\partial K_{zz}}{\partial z}\right)C\right]\\
	+ \frac{1}{2}\left[\frac{\partial^2}{\partial y^2} \left(2K_{yy} C\right) + \frac{\partial^2}{\partial y \partial z} \left(2K_{yz} C\right) + \frac{\partial^2}{\partial z \partial y} \left(2K_{zy} C\right) + \frac{\partial^2}{\partial z^2} \left(2K_{zz} C\right) \right].
\end{multline}
\vspace*{.1cm}
% \begin{equation}
% 	\frac{\partial C}{\partial t} = -\frac{\partial}{\partial y}\left[\left(v+\frac{\partial K_{h}}{\partial y}\right)C\right] -\frac{\partial}{\partial z}\left[\left(w+ \frac{\partial K_{v}}{\partial z}\right)C\right]	+ \frac{1}{2}\left[\frac{\partial^2}{\partial y^2} \left(2K_{h} C\right) + \frac{\partial^2}{\partial z^2} \left(2K_v C\right) \right].
% \end{equation}
This is precisely equation~\eqref{eq:FP-I-ndim} with $\b x =(y,z)$, $p=C$, $\b a = (v+\partial_y K_{yy} + \partial_z K_{yz},\, w+\partial_y K_{zy} + \partial_z K_{zz})$ and $\b D = 2\b K$. Using those notations, $\b x(t) = (y(t),\, z(t))$ obeys thus the Itô SDE
% This is precisely equation~\eqref{eq:FP-I-ndim} with $\b x =(y,z)$, $p=C$, $\b a = (v+\partial_y K_{h},\, w+\partial_z K_{v})$ and $\b D = 2\b K$. Therefore, $\b x(t) = (y(t),\, z(t))$ obeys the Itô SDE
\begin{equation} \label{eq:ito-TM-vec}
	\rm d \b x(t) = \b a(x(t),t) \rm dt + \b B(x(t),t) \rm d \b W(t),
\end{equation}
where $\b B$ has to be solved from $2\b K = \b B \b B^{\t}$. Since $2\b K$ is positive semidefinite, a possible Cholesky decomposition is given by
\begin{equation} \label{eq:B}
	\b B = \begin{pmatrix} B_{yy} & 0 \\ B_{zy} & B_{zz} \end{pmatrix} = \begin{pmatrix} \sqrt{2K_{yy}} & 0 \\ B_* & \sqrt{2K_{zz}-B_*^2} \end{pmatrix},
\end{equation}
where
\begin{equation} \label{eq:Bstar}
	B_* = \left\{ 
		\begin{array}{lr}
			0 & \mbox{if } K_{yy} = 0,\\
			\frac{2K_{yz}}{B_{yy}} & \mbox{otherwise}.
		\end{array}
	\right.
\end{equation}
The Itô SDE~\eqref{eq:ito-TM-vec} can be rewritten as
\begin{subnumcases}{\I\ \label{eq:ito-TM}}
	\rm dy(t) = \left(v + \frac{\partial K_{yy}}{\partial y} + \frac{\partial K_{yz}}{\partial z} \right) \rm dt + B_{yy} \rm dW_1(t)\\
	\rm dz(t) = \left(w + \frac{\partial K_{zy}}{\partial y} + \frac{\partial K_{zz}}{\partial z} \right) \rm dt + B_{zy} \rm dW_1(t) + B_{zz} \rm dW_2(t)\\
	(y(0),z(0)) = (y_0,z_0),
\end{subnumcases}
where $W_1(t)$ and $W_2(t)$ are independent Wiener processes. The derivatives of the elements of $\b K$ appearing in equation~\eqref{eq:ito-TM} are called the \textit{gradient drift terms}. If $\b K$ is known analytically, those terms can be computed for the numerical implementation. If not, finite differences can be used. In the next of this work, we will encounter problems where $\b K$ is discontinuous along segments in the domain, so that a part of the gradient drift terms is infinite at those points.
Such an issue is addressed in \cite{prickett1981random} by neglecting the gradient drift terms all together, and in \cite{tompson1987numerical} by evaluating gradient drift terms via finite differences. Such methods are probably good enough for the simple problems considered in this work.
% \footnote{This is especially true since the discontinuity of the diffusivity is in fact an idealization in the problems we will encounter. An estimation of the gradient drift terms via finite differences near the discontinuities would thus be more realistic than the infinite value.}
However, we prefer the \textit{backward-Itô} approach as this method applies to a wider range of problems with discontinuous diffusivities.

% Specific to overturner :
% In our model, $K_h$ is constant so that $\partial_y K_h = 0$ uniformly on $\Omega$. The term gradient drift term $\partial_z K_v$ is more problematic: $K_v$ is indeed discontinuous and $\partial_z K_v$ is infinite on the segment defined by $(\lambda y_0, z_0)$ with $\lambda \in [0,1]$.


To derive the backward Itô SDE corresponding to the transport model, notice that equation~\eqref{eq:TM} can also be rewritten as
\begin{multline}\label{eq:TM-bito}
	\frac{\partial C}{\partial t} = -\frac{\partial}{\partial y}(vC) -\frac{\partial}{\partial z}(wC) \\+ \frac{1}{2}\left[\frac{\partial}{\partial y} \left(2K_{yy}\frac{\partial C}{\partial y} + 2K_{yz}\frac{\partial C}{\partial z} \right) + \frac{\partial}{\partial z} \left(2K_{zy}\frac{\partial C}{\partial y} + 2K_{zz}\frac{\partial C}{\partial z}\right)\right].
\end{multline}
This is precisely equation~\eqref{eq:FP-bI-ndim} with $\b x = (y,z)$, $p=C$, $\b a = (v,w)$ and $\b D = 2\b K$. Hence, we can use the matrix $\b B$ computed in~\eqref{eq:B}. Then, $\b x(t) = (y(t),z(t))$ also obeys the backward Itô SDE
\begin{subnumcases}{\bI\ \label{eq:bi-TM}}
	\rm dy(t) = v \rm dt + B_{yy}\rm dW_1(t)\\
	\rm dz(t) = w \rm dt + B_{zy}\rm dW_1(t) + B_{zz}\rm dW_2(t)\\
	(y(0),z(0)) = (y_0,z_0).
\end{subnumcases}
Interestingly, there is no gradient drift term in~\eqref{eq:bi-TM}, i.e. no derivative of the diffusivities. In the context of a problem with discontinuous diffusivities, the backward Itô interpretation is thus particularly interesting: see \cite{labolle2000diffusion} and \cite{spivakovskaya2007backward} for more complete discussions about the use of the backward Euler method on problems with discontinuous diffusivities.
%!TEX root = /home/renaud/Documents/EPL/tfe/latex/tfe.tex
\section{The code} \label{sec:thecode}
The preceding sections cover all the material needed to implement a Lagrangian code that solves a two-dimensional advection-diffusion problem. For the need of this work, a \Cpp code has been implemented. The choice of \Cpp is motivated by the fact that it is \textit{fast}, and that it comes together with a wide range of \textit{open source} supporting tools. Another reason is that \Cpp is an \textit{object-oriented language}, and it is widely held that writing in an object-oriented style leads to programs which are easier to understand, to extend, to maintain and to refactor \cite{pitt2012guide}.

The code deals with the two-dimensional transport equation
\begin{equation} \label{eq:TEcode}
	\frac{\partial C}{\partial t} = \nabla \cdot (-\b u C + \b K \nabla C)
\end{equation}
on rectangular domains with no-through boundary conditions. It allows to simulate trajectories, to compute the concentration and to build the transition probability matrix for a given partitioning of the domain. The trajectories are simulated using the backward Euler method, applied on the system of backward Itô SDE's~\eqref{eq:bi-TM}.
% The user can choose between the Euler-Maruyama \textcolor{red}{attention implémenter drift gradient term} and backward-Itô method to simulate the trajectories. In view of the preceding sections, the Itô SDE corresponding to~\eqref{eq:TEcode} is
In this work, only the box-counting method is used for the estimation of the concentration (and thus also for the computation of the transition probability matrix) but the density kernel estimation method has also been implemented for the sake of completeness.

In order to use the code on a particular problem meeting the above specifications, a class that defines the problem must be implemented. That class must inherit from the abstract base class \mintinline{c++}{AbstractAdvDiffProblem} (see listing~\ref{listing:abstractadvdiffproblem}), and must at least implement the two pure virtual functions of the abstract base class:
\begin{listing}[ht!]
\caption{The abstract base class \cppcode{AbstractAdvDiffProblem}.}
\label{listing:abstractadvdiffproblem}
\begin{minted}[breaklines,tabsize=4,fontsize=\footnotesize,style=tango,escapeinside=||]{c++}
class AbstractAdvDiffProblem
{
	protected:
		double mH0, mH1; // boundaries of the domain in the z-direction : H0 <= z <= H1
		double mL0, mL1; // boundaries of the domain in the y-direction : L0 <= y <= L1

	public:
		AbstractAdvDiffProblem(double H0, double H1, double L0, double L1);
		virtual ~AbstractAdvDiffProblem(){};
		double getH0() const; // bottom boundary
		double getH1() const; // top boundary 
		double getL0() const; // left boundary
		double getL1() const; // right boundary
		virtual SymMatrix getK(double y, double z) const=0; // diffusivity tensor
		virtual LowerTriMatrix getB(double y, double z) const; // 2K = BB'
		virtual Vec2 getU(double y, double z) const=0; // velocity vector
		virtual void printInfo(std::ofstream& f) const;
};
\end{minted}
\end{listing}
\begin{itemize}[nosep]
	\item \mintinline[style=tango]{c++}{SymMatrix getK(double y, double z)}: returns the diffusivity tensor $\b K$ evaluated at $(y,z)$. The return value is of type \cppcode{SymMatrix}, a structure intended to store a $2\times 2$ symmetric matrix with only three elements stored in memory. Instantiating an object \cppcode{A} of type \cppcode{SymMatrix} is pretty simple: \cppcode{SymMatrix A(a,b,c)} creates the matrix
	\[
		A = \begin{pmatrix} a & b \\ b & c \end{pmatrix},
	\] 
	where \cppcode{a}, \cppcode{b} and \cppcode{c} are of type double. The elements of \cppcode{A} are then accessed via the syntax \cppcode{A(i,j)} which uses one-based indexing. Hence, \cppcode{A(1,1) = a}, \cppcode{A(1,2) = A(2,1) = b} and \cppcode{A(2,2) = c}. The same syntax can be used to modify the elements of \cppcode{A}.
	\item \mintinline[style=tango]{c++}{Vec2 getU(double y, double z)}: returns the velocity vector $\b u$ evaluated at $(y,z)$. The return value is of type \cppcode{Vec2}, a structure that stores a vector of size 2. The syntax \cppcode{Vec2 v(a,b)} is used to create the two-dimensional vector $v = (a,b)$. The elements of \cppcode{v} are accessed via the syntax \cppcode{v(i)} which also uses one-based indexing.
\end{itemize}
By default, the code computes the matrix $\b B$ using~\eqref{eq:B} and~\eqref{eq:Bstar}. This is done by the function \cppcode{LowerTriMatrix GetB(double y, double z)}. In some cases, it can be interesting to overload that definition of \cppcode{GetB}, which is possible since this function is virtual. The return value must be of type \cppcode{LowerTriMatrix}, which is a structure similar to \cppcode{SymMatrix} but is intended to store lower triangular $2\times2$ matrices instead of symmetric $2\times2$ matrices.

Notice that the code as such implements the dimensional form of the transport model. However, it can be used to run simulations on the adimensional form of the transport model. To this end, it suffice to define the functions \cppcode{getK} and \cppcode{getU} accordingly: \cppcode{getK} shall return the inverse of the Peclet matrix, and \cppcode{getU} shall return the adimensional velocity vector.

Once a class describing the problem is properly defined, three methods can be used to compute either the trajectories, the normalized concentration or the transition probability matrix. We call those methods the \textit{compute methods}. Their signatures are given in listing~\ref{listing:signature_compute}. Since those functions are well documented in the code, we do not provide any further explanations about how to use them here.
\begin{listing}[ht!]
\caption{Signatures of the \textit{compute methods}.}
\label{listing:signature_compute}
\begin{minted}[breaklines,tabsize=4,fontsize=\footnotesize,style=tango,escapeinside=||]{c++}
void ComputeTrajectories(const AbstractAdvDiffProblem& prob, std::string model, double dt,
						 double T, int Nloc, double yStart, double zStart);
void ComputeConcentration(const AbstractAdvDiffProblem &prob, std::string model, double dt,
						  double T, std::string estimator, int Nloc, double yStart,
						  double zStart, int nboxy, int nboxz);
void ComputeTransitionProbabilities(const AbstractAdvDiffProblem& prob, std::string model,
									int nboxy, int nboxz, int nyloc, int nzloc, double dt,
									double Times[], int nTimes, bool binary,
									std::string estimator = "box");
\end{minted}
\end{listing}

Test cases with analytical solutions have been built to assess the validity of the implementation. They are presented in appendix~\ref{app:test_case} and the numerical solution is compared to the analytical solution, producing satisfying results.
% \begin{itemize}
% 	\item \mintinline{c++}{LowerTriMatrix getB(double y, double z)}:
% 	\item io 
% \end{itemize} 
% Le code résout des probleme 2D generaux sur un domaine rectangulaire avec des conditions frontières no trough.
% Choix entre Ito (attention alors il faut implémenter le drift gradient velocity) et backward Ito.
% --> Pour résoudre un problème particulier :
% 	1. Implémenter une class problem qui hérite de abstractadvdiff, et dont les méthode minimales sont : ...
% 	AbstractAdvDiff implémente Cholesky par défaut mais on peut aussi implémenter sa propre méthode getB,
%	ce qui permet d'autre choix de B et dans certains cas une meilleure efficacité. (+ exemple)
%	Si on veut de l'adim il suffit de faire K = 1/Pe et U = ...
%	2. Ensuite les méthodes Compute permette de calculer trajectoire, concentration et matrice de transition de proba.
%	Ces méthodes font entre autre appel aux méthodes de la classe solver et de ses enfant. Par exemple backward ito est implémenté dans 
%	biSlver par <CODE>.
%	3. Mettre le code "main" dans un studycase (à voir si c'est vraiment intéressant de parler des studycases)
% 
% \begin{listing}[ht!]
% \caption{Implementation of the backward Euler method.}
% \label{listing:updateposition}
% \begin{minted}[breaklines,tabsize=4,fontsize=\footnotesize,style=tango,escapeinside=||]{c++}
% void BISolver::UpdatePosition(const AbstractAdvDiffProblem& prob)
% {
% 	LowerTriMatrix B;
% 	Vec2 U;
% 	double R1, R2, dY, dZ, y, z, ypred, zpred;
% 	double sqrt_dt = sqrt(mdt);
% 	for (int i=0; i<mParticles.mN; i++)
% 	{
% 		// position and speed of particle i at time t
% 		y = mParticles.mY[i];
% 		z = mParticles.mZ[i];
% 		U = prob.getU(y,z);
% 		B = prob.getB(y,z);
% 		// realisations of the noises
% 		R1 = wiener(generator);
% 		R2 = wiener(generator);
% 		// prediction step of the backward-Ito scheme
% 		dY = B(1,1)*sqrt_dt*R1;
% 		dZ = B(2,1)*sqrt_dt*R1 + B(2,2)*sqrt_dt*R2;
% 		// No-through BC also applies on the predictions -> bouncing on the walls
% 		ypred = y+dY;
% 		zpred = z+dZ;
% 		ypred = (ypred < prob.getL0()) ? 2*prob.getL0()-|\color{black}{ypred}| : 
% 				(ypred > prob.getL1()) ? 2*prob.getL1()-|\color{black}{ypred}| : ypred;
% 		zpred = (zpred < prob.getH0()) ? 2*prob.getH0()-|\color{black}{zpred}| : 
% 				(zpred > prob.getH1()) ? 2*prob.getH1()-|\color{black}{zpred}| : zpred;
% 		// amplitude of the noises
% 		B = prob.getB(ypred,zpred);
% 		/* update particles positions using backward-Ito scheme
% 		* No-flux BC : Bounce on the wall */
% 		ypred = mParticles.mY[i] + U(1)*mdt + B(1,1)*sqrt_dt*R1;
% 		zpred = mParticles.mZ[i] + U(2)*mdt + B(2,1)*sqrt_dt*R1 + B(2,2)*sqrt_dt*R2;
% 		mParticles.mY[i] = (ypred < prob.getL0()) ? 2*prob.getL0()-|\color{black}{ypred}| : 
% 						   (ypred > prob.getL1()) ? 2*prob.getL1()-|\color{black}{ypred}| : ypred;
% 		mParticles.mZ[i] = (zpred < prob.getH0()) ? 2*prob.getH0()-|\color{black}{zpred}| :
% 						   (zpred > prob.getH1()) ? 2*prob.getH1()-|\color{black}{zpred}| : zpred;
% 	}
% 	mParticles.mTime += mdt;
% }
% \end{minted}
% \end{listing}