%!TEX root = /home/renaud/Documents/EPL/tfe/latex/tfe.tex
\section{Lagrangian equations for the overturner model}
Let us recall the transport equation derived in section \eqref{sec:transport_overturner} for the concentration of a passive tracer in the \textit{overturner} model:
\begin{equation} \label{eq:TM}
	\frac{\partial C}{\partial t} = \nabla \cdot (-\b u C + \b K \nabla C).
\end{equation}
This equation can be interpreted as a Fokker-Planck equation where $C$ is the probability density function of the position $\b x(t) = (y(t),\,z(t))$ of the particle. 
% For the sake of generality, let us first consider a general symmetric and positive-definite diffusivity tensor
% \begin{equation}
% 	\b K = \begin{pmatrix} K_{yy} & K_{yz}\\ K_{zy} & K_{zz} \end{pmatrix},
% \end{equation}
% where $K_{yz} = K_{zy}$ since $\b K$ is symmetric. Equation \eqref{eq:TM} can be rewritten as
% \begin{align}\label{eq:TM-bito}
% 	\frac{\partial C}{\partial t} =~& -\frac{\partial}{\partial y}\left[\left(v+\frac{\partial K_{yy}}{\partial y} + \frac{\partial K_{yz}}{\partial z}\right)C\right] -\frac{\partial}{\partial z}\left[\left(w+\frac{\partial K_{zy}}{\partial y} + \frac{\partial K_{zz}}{\partial z}\right)C\right] \nonumber\\
% 	&+ \frac{1}{2}\left[\frac{\partial^2}{\partial y^2} \left(2K_{yy} C\right) + \frac{\partial^2}{\partial y \partial z} \left(2K_{yz} C\right) + \frac{\partial^2}{\partial z \partial y} \left(2K_{zy} C\right) + \frac{\partial^2}{\partial z^2} \left(2K_v C\right) \right].
% \end{align}
% This is precisely equation \eqref{eq:FP-I-ndim} with $\b x =(y,z)$, $p=C$, $\b a = (v+\partial_y K_{yy} + \partial_z K_{yz},\, w+\partial_y K_{zy} + \partial_z K_{zz})$ and $\b D = 2\b K$. Therefore, $\b x(t) = (y(t),\, z(t))$ obeys the Itô SDE
Equation \eqref{eq:TM} can be rewritten as
\begin{equation}
	\frac{\partial C}{\partial t} = -\frac{\partial}{\partial y}\left[\left(v+\frac{\partial K_{h}}{\partial y}\right)C\right] -\frac{\partial}{\partial z}\left[\left(w+ \frac{\partial K_{zz}}{\partial z}\right)C\right]	+ \frac{1}{2}\left[\frac{\partial^2}{\partial y^2} \left(2K_{yy} C\right) + \frac{\partial^2}{\partial z^2} \left(2K_v C\right) \right].
\end{equation}
This is precisely equation \eqref{eq:FP-I-ndim} with $\b x =(y,z)$, $p=C$, $\b a = (v+\partial_y K_{h},\, w+\partial_z K_{v})$ and $\b D = 2\b K$. Therefore, $\b x(t) = (y(t),\, z(t))$ obeys the Itô SDE
\begin{equation} \label{eq:ito-overturner-vec}
	\rm d \b x(t) = \b a(x(t),t) \rm dt + \b B(x(t),t) \rm d \b W(t),
\end{equation}
where $\b B$ has to be solved from $2\b K = \b B \b B^{\t}$. Since $2\b K$ is positive definite, the unique Cholesky decomposition yields
\begin{equation} \label{eq:B}
	\b B = \begin{pmatrix} \sqrt{2K_{h}} & 0 \\ 0 & \sqrt{2K_{v}} \end{pmatrix},
\end{equation}
and the Itô SDE \eqref{eq:ito-overturner-vec} can be rewritten as
\begin{equation} \label{eq:ito-overturner}
	\begin{cases}
		\rm dy(t) = \left(v + \frac{\partial K_h}{\partial y} \right) \rm dt + \sqrt{2K_h}\rm dW_1(t)\\
		\rm dz(t) = \left(w + \frac{\partial K_v}{\partial z} \right) \rm dt + \sqrt{2K_v}\rm dW_2(t)\\
		(y(0),z(0)) = (y_0,z_0),
	\end{cases}
\end{equation}
where $W_1(t)$ and $W_2(t)$ are independent Wiener processes. In our model, $K_h$ is constant so that $\partial_y K_h = 0$ uniformly on $\Omega$. The term gradient drift term $\partial_z K_v$ is more problematic: $K_v$ is indeed discontinuous and $\partial_z K_v$ is infinite on the segment defined by $(\lambda y_0, z_0)$ with $\lambda \in [0,1]$. Such a problem is addressed in \cite{prickett1981random} by neglecting the gradient drift terms all together, and in \cite{tompson1987numerical} by evaluating gradient drift terms via finite differences. Such methods are probably good enough for our simple overturner model.\footnote{This is especially through since the discontinuity of the vertical diffusivity is an idealization of the reality. An estimation of $\partial_z K_v$ via finite differences near the discontinuities would thus be more realistic than the infinite value.} However, we prefer the \textit{backward-Euler} approach as this method applies to a wider range of problems with discontinuous diffusivities.

To derive the backward Itô SDE corresponding to the overturner transport model, notice that equation \eqref{eq:TM} can also be rewritten as
\begin{equation}\label{eq:TM-bito}
	\frac{\partial C}{\partial t} = -\frac{\partial}{\partial y}(vC) -\frac{\partial}{\partial z}(wC) + \frac{1}{2}\left[\frac{\partial}{\partial y} \left(2K_h\frac{\partial C}{\partial y} \right) + \frac{\partial}{\partial z} \left(2K_v(y,z)\frac{\partial C}{\partial z}\right)\right].
\end{equation}
This is precisely equation \eqref{eq:FP-bI-ndim} with $\b x = (y,z)$, $p=C$, $\b a = (v,w)$ and $\b D = 2\b K$. Hence, we can use the matrix $\b B$ computed in \eqref{eq:B}. Then, $\b x(t) = (y(t),z(t))$ also obeys the backward Itô SDE
\begin{equation} \label{eq:bi-overturner}
	\begin{cases}
		\rm dy(t) = v \rm dt + \sqrt{2K_h}\rm dW_1(t)\\
		\rm dz(t) = w \rm dt + \sqrt{2K_v}\rm dW_2(t)\\
		(y(0),z(0)) = (y_0,z_0).
	\end{cases}
\end{equation}
Interestingly, there is no drift term, i.e. no derivative of the diffusivities. In the context of a problem with discontinuous diffusivities, the backward Itô interpretation is thus particularly interesting: see \cite{labolle2000diffusion} and \cite{spivakovskaya2007backward} for more complete discussions about the use of backward Itô method on problem with discontinuous diffusivities.