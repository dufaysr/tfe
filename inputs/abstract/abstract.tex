%!TEX root = /home/renaud/Documents/EPL/tfe/latex/tfe.tex
\addcontentsline{toc}{section}{Abstract}
\begin{abstract}
	In the context of solving marine problems, compartment models act as a complementary tool to the usual discretization method such as finite difference or finite element method: although they provide less accurate results, they allows for easier interpretation and faster simulation, hence providing the possibility for long-term model runs. However, the compartment model approach suffers from the fact that there is nowadays no automatic procedure to delineate the compartments and express the fluxes between them. An method towards automatic delineation of the compartments. It is based on a community detection method from networks theory, the stability method. The whole procedure is applied on a simple problem, showing satisfying results. Obviously, the problem considered is too simple to draw conclusions on the method proposed, but the encouraging result in this work shows that further work is worthwhile.

	Alongside the presentations of the method and of the stability criterion for community detection, a whole chapter is dedicated to the theory of stochastic differential equations, leading to a Lagrangian numerical method for solving a two dimensional advection-diffusion problem. Besides, a generic compartment model is build from the reactive transport equation, and its fundamental properties are shown. The only theoretical background needed to understand this work is to be familiar with the reactive transport equation. All the other theoretical tools are introduced and proved before they are used, making this work a standalone document. 
	% The great results of community detection methods from network theory in many complex research fields such as biology, sociology or epidemiology suggest 
\end{abstract}