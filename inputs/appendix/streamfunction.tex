%!TEX root = /home/renaud/Documents/EPL/tfe/latex/tfe.tex
\chapter{Applications}
\section{The streamfunction} \label{app:streamfunction}
The interest of the streamfunction comes from the following property.
\begin{property} \label{prop:streamfunction}
	For any 3-dimensional potential vector $\b a(x,y,z)$ whose second partial derivatives are continuous,
	\begin{equation}
		\nabla \cdot (\nabla \times \b a) = 0.
	\end{equation}
\end{property}
\begin{proof}
	The proof is quite straightforward :
	\begin{align*}
		\nabla \cdot (\nabla \times \b a(x,y,z)) &= \nabla \cdot \left[\left(\frac{\partial a_z}{\partial y}-\frac{\partial a_y}{\partial z}\right)\ex + \left(\frac{\partial a_x}{\partial z}-\frac{\partial a_z}{\partial x}\right)\ey + \left(\frac{\partial a_y}{\partial x}-\frac{\partial a_x}{\partial y}\right)\ez \right]\\
		&= \frac{\partial^2 a_z}{\partial x \partial y} - \frac{\partial^2 a_y}{\partial x \partial z} + \frac{\partial^2 a_x}{\partial y \partial z} - \frac{\partial^2 a_z}{\partial y \partial x} + \frac{\partial^2 a_y}{\partial z \partial x} - \frac{\partial^2 a_x}{\partial z \partial y}\\
		&= 0.  
	\end{align*}
	The assumption on the continuity of the second partial derivatives allows to use \textit{Schwarz's theorem} which states that the second partial derivatives are symmetric.
\end{proof}
This property suggests a simple method to build a divergence-free velocity field, namely one that satisfies the continuity equation \eqref{eq:continuity_boussinesq}. Indeed, it is sufficient to choose a relevant three-dimensional vector $\b a(x,y,z) = (a_x(x,y,z),\,a_y(x,y,z),\,a_z(x,y,z))$ with $a_x$, $a_y$ and $a_z$ of class $\C^2$ and impose that 
\begin{equation} \label{eq:stream3d}
	\b u = \begin{pmatrix} u \\ v \\ w \end{pmatrix} = - \nabla \times \b a = - \begin{pmatrix} \frac{\partial a_z}{\partial y}-\frac{\partial a_y}{\partial z}\\[.1 cm]
													\frac{\partial a_x}{\partial z}-\frac{\partial a_z}{\partial x}\\[.1 cm]
													\frac{\partial a_y}{\partial x}-\frac{\partial a_x}{\partial y}
									\end{pmatrix}.
\end{equation}
This ensures that $\b u$ is divergence-free since $\nabla \cdot \b u = - \div \p{\rot \b a} = 0$.

In this work, we restrict ourselves to two-dimensional flows in the plane. Let us consider the plane $(\ey,\ez)$ without loss of generality. In that case, $u = 0$ and $\partial \cdot/\partial x = 0$. The relation \eqref{eq:stream3d} becomes 
\begin{equation}
	\b u = \begin{pmatrix} 0 \\ v \\ w \end{pmatrix} = - \nabla \times \b a = \begin{pmatrix} 0\\ - \frac{\partial a_x}{\partial z}\\[.1 cm] \frac{\partial a_x}{\partial y} \end{pmatrix},
\end{equation}
where only the component $a_x$ is needed to describe $\b u$. Hence, the velocity field of a flow in the plane is described by a scalar quantity, the so-called \textit{streamfunction}, generally noted $\psi$. The potential vector $\b a$ is thus of the form $\b a(y,z) = (\psi(y,z),\, 0,\, 0)$, and the components of the velocity vector are given by :
\begin{equation} \label{eq:u-psi}
	v = -\frac{\partial \psi}{\partial z}, \quad w = \frac{\partial \psi}{\partial y}.
\end{equation}
Note that adding any constant to $\psi$ leaves the velocity vector unchanged. This adds some freedom to the choice of $\psi$. In order to derive a streamfunction that is relevant to some particular problem, it is of top-notch interest to get some physical intuition about the streamfunction. To this end, two fundamental properties of the streamfunction in two dimensions are shown in the next paragraphs. 

First, notice that assuming that $\psi \in \C^2$ implies straightforwardly that $\rm d\psi = (\partial \psi/\partial y)\rm dy + (\partial \psi/\partial z) \rm dz$ is an \textit{exact differential} since by Schwarz's theorem
\begin{equation}
	\frac{\partial^2 \psi}{\partial y \partial z} = \frac{\partial^2 \psi}{\partial z \partial y}.
\end{equation}
Thus,
\begin{equation}
\int_{\b x_1}^{\b x_2} \rm d\psi = \psi(y_2,z_2) - \psi(y_1,z_1)	
\end{equation}
is path-independent.

\begin{property} \label{prop:streamfunction-tangent}
	In two dimensions, the curves along which $\psi$ is constant (i.e. the \textit{isolines} of  $\psi$) are exactly the \textit{streamlines} of the flow, namely the family of curves that are instantaneously tangent to the velocity vector.
\end{property}
\begin{proof}
	Let an isoline of $\psi$ be parametrized by $s \mapsto \b x_S(s) = (y_S(s),z_S(s))$. The fact that $\psi$ is constant along that curve implies that $\rm d\psi_S = (\partial \psi/\partial y)\rm dy_S + (\partial \psi/\partial z) \rm dz_S = \nabla \psi \cdot \rm d \b x_S = 0$. This shows that the vector $\nabla \psi$ is normal to the curve $\b x_S(s)$. Hence, showing that $\b x_S(s)$ is everywhere tangent to $\b u$ is equivalent to showing that $\b u \cdot \nabla \psi = 0$ everywhere. The latter is straightforward using relation \eqref{eq:u-psi} :
\begin{equation}
	\b u \cdot \nabla \psi = -\frac{\partial \psi}{\partial z}\frac{\partial \psi}{\partial y} + \frac{\partial \psi}{\partial y}\frac{\partial \psi}{\partial z} = 0,
\end{equation}
which concludes the proof.
\end{proof}

\begin{property} \label{prop:streamfunction-flowrate}
	In two dimensions, the \textit{volume flow rate} between two streamlines of values $\psi_1$ and $\psi_2$ is equal to the difference of those streamlines, $\psi_1 - \psi_2$.
\end{property}
\begin{proof}
	Consider two infinitely close points $\b x_1 = (y_1,\, z_1)$ and $\b x_1 + \rm d \b x = (y_1 + \rm dy,\, z_1 + \rm dz)$. At those points, the streamfunction has values $\psi(y_1,z_1) = \psi_1$ and $\psi(y_1+\rm dy,z_1 + \rm dz) = \psi_1 + \rm d\psi$. Let us now consider the volume flow rate $\rm dq$ across the infinitesimal segment $[\b x_1,\, \b x_1 + \rm d \b x]$, positive in the right-hand side direction of the segment if the latter is directed from $\b x_1$ to $\b x_1 + \rm d \b x$. It is equal to $\b u \cdot \hat{\b{n}}$, where $\hat{\b{n}} = (\rm dz,\,-\rm dy)$ is the unit vector normal to the segment, oriented in the right-hand side direction. Hence, $\rm dq = v \rm dz - w \rm dy$, which, using relation \eqref{eq:u-psi}, amounts to
	\begin{equation}
		\rm dq = - \rm d\psi.
	\end{equation}
	Now, consider any two points $\b x_1 = (y_1, z_1)$ and $\b x_2 = (y_2, z_2)$ in the (connected) domain. The volume flow rate $q_{1 \rightarrow 2}$ across any curve $\gamma_{1 \rightarrow 2}$ connecting $\b x_1$ to $\b x_2$, positive in the right-hand side direction of the directed segment $[\b x_1,\, \b x_2]$ is
	\begin{equation}
		q_{1 \rightarrow 2} = \int_{\gamma_{1 \rightarrow 2}} \rm dq = \int_{\b x_1}^{\b x_2} (-\rm d\psi) = \psi(y_1,z_1) - \psi(y_2,z_2), 	
	\end{equation} 
	where $\int_{\gamma_{1 \rightarrow 2}}$ is the line integral along a curve connecting $\b x_1$ to $\b x_2$, afterwards noted $\int_{\b x_1}^{\b x_2}$ to emphasize the fact that it does not depend on the integration path, since $\rm d\psi$ is an exact differential.
\end{proof}